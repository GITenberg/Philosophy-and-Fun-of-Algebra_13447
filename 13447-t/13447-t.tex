%%%%%%%%%%%%%%%%%%%%%%%%%%%%%%%%%%%%%%%%%%%%%%%%%%%%%%%%%%%%%%%%%%%%%%%%%%%
%% Project Gutenberg's Philosophy and Fun of Algebra, by Mary Everest    %%
%% Boole                                                                 %%
%%                                                                       %%
%% This eBook is for the use of anyone anywhere at no cost and with      %%
%% almost no restrictions whatsoever.  You may copy it, give it away or  %%
%% re-use it under the terms of the Project Gutenberg License included   %%
%% with this eBook or online at www.gutenberg.net                        %%
%%                                                                       %%
%%                                                                       %%
%% Packages and substitutions:                                           %%
%%                                                                       %%
%% book:     Basic book class. Required.                                 %%
%% amsmath:  Basic AMS math. Required.                                   %%
%% amssymb:  Basic AMS symbols. Required.                                %%
%% inputenc: Basic Accept different input encodings.                     %%
%%           Could be dispensed with by changing all                     %%
%%           ISO-8859-1-specific characters.                             %%
%% graphicx: Basic graphics for images. Required.                        %%
%% yfonts:   Support for old German fonts                                %%
%%           Used for one word in \textgoth                              %%
%%                                                                       %%
%% Producer's Comments:                                                  %%
%%                                                                       %%
%%                                                                       %%
%%                                                                       %%
%% Things to Check:                                                      %%
%%                                                                       %%
%% Spellcheck: OK                                                        %%
%% LaCheck: OK                                                           %%
%% Lprep/gutcheck: OK                                                    %%
%% PDF pages, excl. Gutenberg boilerplate:  47                           %%
%% PDF pages, incl. Gutenberg boilerplate:  56                           %%
%% ToC page numbers: OK                                                  %%
%% Images: One: cornell.png                                              %%
%% Fonts: OK. Note \textgoth ygoth.pfb for one word                      %%
%% Advertisement \fboxes: OK                                             %%
%% The first Advertisement page, "Logic Taught by Love", page 45 in      %%
%% the PDF as compiled Dec 05, is uncomfortably long, with an overfull   %%
%% \vbox.                                                                %%
%%                                                                       %%
%%                                                                       %%
%% Compile History:                                                      %%
%%                                                                       %%
%% Dec 05: jt. Compiled with                                             %%
%%             pdflatex 13447-t                                          %%
%%             pdflatex 13447-t                                          %%
%%             pdflatex 13447-t                                          %%
%%                                                                       %%
%%                                                                       %%
%%                                                                       %%
%%%%%%%%%%%%%%%%%%%%%%%%%%%%%%%%%%%%%%%%%%%%%%%%%%%%%%%%%%%%%%%%%%%%%%%%%%%
\documentclass[oneside]{book}
\listfiles
\usepackage[latin1]{inputenc}
\usepackage{amsmath}
\usepackage{graphicx}
\usepackage{yfonts}

\begin{document}

\thispagestyle{empty}
\small
\begin{verbatim}
Project Gutenberg's Philosophy and Fun of Algebra, by Mary Everest
Boole

This eBook is for the use of anyone anywhere at no cost and with
almost no restrictions whatsoever.  You may copy it, give it away or
re-use it under the terms of the Project Gutenberg License included
with this eBook or online at www.gutenberg.net


Title: Philosophy and Fun of Algebra

Author: Mary Everest Boole

Release Date: September 12, 2004 [EBook #13447]
[Date last updated: December 3, 2005]

Language: English

Character set encoding: TeX

*** START OF THIS PROJECT GUTENBERG EBOOK PHILOSOPHY AND FUN OF ALGEBRA ***




Produced by Joshua Hutchinson, John Hagerson, and the Project
Gutenberg On-line Distributed Proofreaders. This book was produced
from images provided by Cornell University.






\end{verbatim}
\normalsize
\newpage

\frontmatter

\bigskip\bigskip\bigskip
\begin{center}
\Huge PHILOSOPHY \& FUN \\
OF ALGEBRA \\

\bigskip\bigskip
\normalsize BY \\
\Large MARY EVEREST BOOLE \\
\bigskip
\footnotesize AUTHOR OF \\
``PREPARATION OF THE CHILD FOR SCIENCE,'' ETC. \\

\bigskip\bigskip\bigskip\bigskip
\normalsize LONDON: C.~W.~DANIEL, LTD. \\
3 Tudor Street, E.C.~4.
\end{center}

\newpage

\begin{center}
\textbf{Production Note}
\end{center}

Cornell University Library produced this volume to replace the
irreparably deteriorated original. It was scanned using Xerox
software and equipment at 600 dots per inch resolution and
compressed prior to storage using CCITT Group 4 compression. The
digital data were used to create Cornell's replacement volume on
paper that meets the ANSI Standard Z39.48-1984. The production of
this volume was supported in part by the Commission on Preservation
and Access and the Xerox Corporation. 1990.

\bigskip\bigskip\bigskip\bigskip

\begin{center}
\includegraphics[width=75mm]{images/cornell.png} \\

\bigskip \footnotesize BOUGHT WITH THE INCOME OF THE \\
\smallskip \normalsize SAGE ENDOWMENT FUND \\
\medskip \footnotesize THE GIFT OF \\
\smallskip \normalsize HENRY W. SAGE \\
\bigskip 1891
\end{center}

\newpage

\begin{center}
\fbox{\parbox{11cm}{
\begin{center}
\textsc{Works by} \\
MARY EVEREST BOOLE \\
\medskip \rule{4cm}{1pt} \\
\smallskip \textsc{Logic Taught By Love.} 3s.\ 6d.\ net. \\
\smallskip\textsc{Mathematical Psychology of Gratry and \\
Boole for Medical Students.} 3s.\ 6d.\ net. \\
\smallskip\textsc{Boole's Psychology as a Factor in Education.}
   6d.\ net. \\
\smallskip\textsc{The Message of Psychic Science to the World.}
   3s.\ 6d.\ net. \\
\smallskip\textsc{Mistletoe and Olive.} 1s.\ 6d.\ net. \\
\smallskip\textsc{Miss Education and Her Garden.} 6d.\ net. \\
\smallskip\textsc{Philosophy and Fun of Algebra.} 2s.\ net. \\
\medskip C.W.\ DANIEL. \\
\rule{4cm}{1pt} \\
\smallskip\smallskip \textsc{The Preparation of the Child
   for Science.} 2s. \\
\smallskip\textsc{The Logic of Arithmetic.} 2s. \\
\medskip CLARENDON PRESS.
\end{center}}}
\end{center}

\newpage

\begin{center}
\large \textgoth{To} \\
\bigskip \textsc{BASIL and MARGARET} \\
\rule{4cm}{1pt}
\end{center}

\textsc{My Dear Children,}

A young monkey named Genius picked a green walnut, and bit, through
a bitter rind, down into a hard shell. He then threw the walnut
away, saying: ``How stupid people are! They told me walnuts are good
to eat.''

His grandmother, whose name was Wisdom, picked up the
walnut---peeled off the rind with her fingers, cracked the shell,
and shared the kernel with her grandson, saying: ``Those get on best
in life who do not trust to first impressions.''

In some old books the story is told differently; the grandmother is
called Mrs Cunning-Greed, and she eats all the kernel herself.
Fables about the Cunning-Greed family are written to make children
laugh. It is good for you to laugh; it makes you grow strong, and
gives you the habit of understanding jokes and not being made
miserable by them. But take care not to believe such fables;
because, if you believe them, they give you bad dreams.

\medskip \hfill MARY EVEREST BOOLE.

\emph{January} 1909.

\tableofcontents

%% CONTENTS
%%
%% CHAP.
%%  1. FROM ARITHMETIC TO ALGEBRA
%%  2. THE MAKING OF ALGEBRAS
%%  3. SIMULTANEOUS PROBLEMS
%%  4. PARTIAL SOLUTIONS AND THE PROVISIONAL
%%     ELIMINATION OF ELEMENTS OF COMPLEXITY
%%  5. MATHEMATICAL CERTAINTY AND REDUCTIO
%%     AD ABSURDUM
%%  6. THE FIRST HEBREW ALGEBRA
%%  7. HOW TO CHOOSE OUR HYPOTHESES
%%  8. THE LIMITS OF THE TEACHER'S FUNCTION
%%  9. THE USE OF SEWING CARDS
%% 10. THE STORY OF A WORKING HYPOTHESIS
%% 11. MACBETH'S MISTAKE
%% 12. JACOB'S LADDER
%% 13. THE GREAT \emph{X} OF THE WORLD
%% 14. GO OUT OF MY CLASS-ROOM
%% 15. $\sqrt{-1}$
%% 16. INFINITY
%% 17. FROM BONDAGE TO FREEDOM
%% APPENDIX

\mainmatter

\chapter{From Arithmetic To Algebra}

Arithmetic means dealing logically with facts which we know (about
questions of number).

``Logically''; that is to say, in accordance with the ``Logos'' or
hidden wisdom, \textit{i.e.} the laws of normal action of the human
mind.

For instance, you are asked what will have to be paid for six pounds
of sugar at 3d.\ a pound. You multiply the six by the three. That is
not because of any property of sugar, or of the copper of which the
pennies are made. You would have done the same if the thing bought
had been starch or apples. You would have done just the same if the
material had been tea at 3s.\ a pound. Moreover, you would have done
just the same \textit{kind} of action if you had been asked the
price of seven pounds of tea at 2s.\ a pound. You do what you do
under direction of the Logos or hidden wisdom. And this law of the
Logos is made not by any King or Parliament, but by whoever or
whatever created the human mind. Suppose that any Parliament passed
an act that all the children in the kingdom were to divide the price
by the number of pounds; the Parliament could not make the answer
come right. The only result of a foolish Act of Parliament like that
would be that everybody's sums would come wrong, and everybody's
accounts be in confusion, and everybody would wonder why the trade
of the country was going to the bad.

In former times there were kings and emperors quite stupid enough to
pass Acts like that, but governments have grown wiser by experience
and found out that, as far as arithmetic goes, there is no use in
ordering people to go contrary to the laws of the Logos, because the
Logos has the whip hand, and knows its own business, and is master
of the situation. Therefore children now are allowed to study the
laws of the Logos, whenever the business on hand is finding out how
much they are to pay in a shop.

Sometimes your teachers set you more complicated problems
than:---What is the price of six pounds of sugar? For instance:---In
what proportion must one mix tea bought at 1s.\ 4d.\ a pound with
tea bought at 1s.\ 10d.\ a pound so as to make 5 per cent.\ profit
by selling the mixture at 1s.\ 9d.\ a pound?

Arithmetic, then, means dealing logically with certain facts that we
know, about number, with a view to arriving at knowledge which as
yet we do not possess.

When people had only arithmetic and not algebra, they found out a
surprising amount of things about numbers and quantities. But there
remained problems which they very much needed to solve and could
not. They had to guess the answer; and, of course, they usually
guessed wrong. And I am inclined to think they disagreed. Each
person, of course, thought his own guess was nearest to the truth.
Probably they quarrelled, and got nervous and overstrained and
miserable, and said things which hurt the feelings of their friends,
and which they saw afterwards they had better not have said---things
which threw no light on the problem, and only upset everybody's mind
more than ever. I was not there, so I cannot tell you exactly what
happened; but quarrelling and disagreeing and nerve-strain always do
go on in such cases.

At last (at least I should suppose this is what happened) some man,
or perhaps some woman, suddenly said: ``How stupid we've all been!
We have been dealing logically with all the facts we knew about this
problem, except the most important fact of all, the fact of our own
ignorance. Let us include that among the facts we have to be logical
about, and see where we get to then. In this problem, besides the
numbers which we do know, there is one which we do not know, and
which we want to know. Instead of guessing whether we are to call it
nine, or seven, or a hundred and twenty, or a thousand and fifty,
let us agree to call it $x$, and let us always remember that $x$
stands for the Unknown. Let us write $x$ in among all our other
numbers, and deal logically with it according to exactly the same
laws as we deal with six, or nine, or a hundred, or a thousand.''

As soon as this method was adopted, many difficulties which had been
puzzling everybody fell to pieces like a Rupert's drop when you nip
its tail, or disappeared like bats when the sun rises. Nobody knew
where they had gone to, and I should think that nobody cared. The
main fact was that they were no longer there to puzzle people.

A little girl was once saying the Evening Hymn to me, ``May no ill
dreams disturb my rest, No powers of darkness me molest.'' I asked
if she knew what \textit{Powers of Darkness} meant. She said, ``The
wolves which I cannot help fancying are under my bed when all the
time I know they are not there. They must be the Powers of Darkness,
because they go away when the light comes.''

Now that is exactly what happened when people left off disputing
about what they did not know, and began to deal logically with the
fact of their own ignorance. This method of solving problems by
honest confession of one's ignorance is called
Algebra.\footnote{\textit{See} Appendix.}

The name Algebra is made up of two Arabic words.

The science of Algebra came into Europe through Arabs, and therefore
is called by its Arabic name. But it is believed to have been known
in India before the Arabs got hold of it.

Any fact which we know or have been told about our problem is called
a datum. The number of pounds of sugar we are to buy is one datum;
the price per pound is another.

The plural of datum is data. It is a good plan to write all one's
data on one column or page of the paper and work one's sum on the
other. This leaves the first column clear for adding to one's data
if one finds out any fresh one.

\chapter{The Making of Algebras}

The Arabs had some cousins who lived not far off from Arabia and who
called themselves Hebrews. A taste for Algebra seems to have run in
the family. Three Algebras grew up among the Hebrews; I should think
they are the grandest and most useful that ever were heard of or
dreamed of on earth.

One of them has been worked into the roots of all our science; the
second is much discussed among persons who have leisure to be very
learned. The third has hardly yet begun to be used or understood in
Europe; learned men are only just beginning to think about what it
really means. All children ought to know about at least the first of
these.

But, before we begin to talk about the Hebrew Algebras, there are
two or three things that we must be quite clear about.

Many people think that it is impossible to make Algebra about
anything except number. This is a complete mistake. We make an
Algebra whenever we arrange facts that we know round a centre which
is a statement of what it is that we want to know and do not know;
and then proceed to deal logically with all the statements,
including the statement of our own ignorance.

Algebra can be made about anything which any human being wants to
know about. Everybody ought to be able to make Algebras; and the
sooner we begin the better. It is best to begin before we can talk;
because, until we can talk, no one can get us into illogical habits;
and it is advisable that good logic should get the start of bad.

If you have a baby brother, it would be a nice amusement for you to
teach him to make Algebra when he is about ten months or a year old.
And now I will tell you how to do it.

Sometimes a baby, when it sees a bright metal tea-pot, laughs and
crows and wants to play with the baby reflected in the metal. It has
learned, by what is called ``empirical experience,'' that tea-pots
are nice cool things to handle. Another baby, when it sees a bright
tea-pot, turns its head away and screams, and will not be pacified
while the tea-pot is near. It has learned, by empirical experience,
that tea-pots are nasty boiling hot things which burn one's fingers.

Now you will observe that both these babies have learnt by
experience. Some people say that experience is the mother of Wisdom;
but you see that both babies cannot be right; and, as a matter of
fact, both are wrong. If they could talk, they might argue and
quarrel for years; and vote; and write in the newspapers; and waste
their own time and other people's money; each trying to prove he was
right. But there is no wisdom to be got in that way. What a wise
baby knows is that he \emph{cannot tell}, by the mere look of a
tea-pot, whether it is hot or cold. The fact that is most prominent
in his mind when he sees a tea-pot is the fact that \emph{he does
not know} whether it is hot or cold. He puts that fact along with
the other fact:---that he would very much like to play with the
picture in the tea-pot supposing it would not burn his fingers; and
he deals logically with both these facts; and comes to the wise
conclusion that it would be best to go very cautiously and find out
whether the tea-pot is hot, by putting his fingers near, but not too
near. That baby has begun his mathematical studies; and begun them
at the right end. He has made an Algebra for himself. And the best
wish one can make for his future is that he will go on doing the
same for the rest of his life.

Perhaps the best way of teaching a baby Algebra would be to get him
thoroughly accustomed to playing with a bright vessel of some kind
when cold; then put it and another just like it on the table in
front of him, one being filled with hot water. Let him play with the
cold one; and show him that you do not wish him to play with the
other. When he persists, as he probably will, let him find out for
himself that the two things which look so alike have not exactly the
same properties. Of course, you must take care that he does not hurt
himself seriously.

\chapter{Simultaneous Problems}

It often happens that two or three problems are so entangled up
together that it seems impossible to solve any one of them until the
others have been solved. For instance, we might get out three
answers of this kind:---
\begin{center}\begin{tabular}{c}
  $x$ equals half of $y$; \\
  $y$ equals twice $x$; \\
  $z$ equals $x$ multiplied by $y$.
\end{tabular}\end{center}
The value of each depends on the value of the others.

When we get into a predicament of this kind, three courses are open
to us.

We can begin to make slap-dash guesses, and each argue to prove that
his guess is the right one; and go on quarrelling; and so on; as I
described people doing about arithmetic before Algebra was invented.

Or we might write down something of this kind:---

The values cannot be known. There is no answer to our problem.

We might write:---
\begin{center}\begin{tabular}{c}
  $x$ is the unknowable; \\
  $y$ is non-existent; \\
  $z$ is imaginary,
\end{tabular}\end{center}
and accept those as answers and give them forth to the world with
all the authority which is given by big print, wide margins, a
handsome binding, and a publisher in a large way of business; and so
make a great many foolish people believe we are very wise.

Some people call this way of settling things Philosophy; others call
it arrogant conceit. Whatever it is, it is not Algebra. The Algebra
way of managing is this:---

We say: Suppose that $x$ were Unity (1); what would become of $y$
and $z$? Then we write out our problem as before; only that,
wherever there was $x$, we now write 1.

If the result of doing so is to bring out some such ridiculous
answer as ``2 and 3 make 7,'' we then know that $x$ cannot be 1. We
now add to our column of data, ``$x$ cannot be 1.''

But if we come to a truism, such as ``2 and 3 make 5,'' we add to
our column of data, ``$x$ may be 1.'' Some people add to their
column of data, ``$x$ is 1,'' but that again is not Algebra. Next we
try the experiment of supposing $x$ to be equal to zero (0), and go
over the ground again.

Then we go over the same ground, trying $y$ as 1 and as 0.

And then we try the same with $z$. Some people think that it is
waste of time to go over all this ground so carefully, when all you
get by it is either nonsense, such as ``2 and 3 are 7''; or truisms,
such as ``2 and 3 are 5.'' But it is not waste of time. For, even if
we never arrive at finding out the value of $x$, or $y$, or $z$,
every conscientious attempt such as I have described adds to our
knowledge of the structure of Algebra, and assists us in solving
other problems.

Such suggestions as ``suppose $x$ were Unity'' are called ``working
hypotheses,'' or ``hypothetical data.'' In Algebra we are very
careful to distinguish clearly between actual data and hypothetical
data.

This is only part of the essence of Algebra, which, as I told you,
consists in preserving a constant, reverent, and conscientious
awareness of our own ignorance.

When we have exhausted all the possible hypotheses connected with
Unity and Zero, we next begin to experiment with other values of
$x$; \emph{e.g.}---suppose $x$ were 2, suppose $x$ were 3, suppose
it were 4. Then, suppose it were one half, or one and a half, and so
on, registering among our data, each time, either ``$x$ may be so
and so,'' or ``$x$ cannot be so and so.''

The method of finding out what $x$ cannot be, by showing that
certain suppositions or hypotheses lead to a ridiculous statement,
is called the method of \emph{reductio ad absurdum}. It is largely
used by Euclid.

\chapter[Partial Solutions\ldots\/Elements of Complexity]{Partial Solutions
and the Provisional Elimination of Elements of Complexity}

Suppose that we never find out for certain whether $x$ is unity or
zero or something else, we then begin to experiment in a different
direction. We try to find out which of the hypothetical values of
$x$ throw most light on other questions, and if we find that some
particular value of $x$---for instance, unity---makes it easier than
does any other value to understand things about $y$ and $z$, we have
to be very careful not to slip into asserting that $x$ \emph{is}
unity. But the teacher would be quite right in saying to the class,
``For the present we will leave alone thinking about what would
happen if $x$ were something different from unity, and attend only
to such questions as can be solved on the supposition that $x$ is
unity.'' This is what is called in Algebra ``provisional elimination
of some elements of complexity.''

It might happen that one of the older pupils, specially clever at
mathematics, but not very well disciplined, should start some point
connected with the supposition that $x$ is something different than
unity. It would be the teacher's business to remind her: ``At
present we are dealing with the supposition that $x$ \emph{is}
unity. When we have exhausted that subject we will investigate your
question. But, till then, please do not distract the attention of
the class by talking about what is not the business on hand at
present.''

If the girl forgot, the teacher might say: ``I should very much like
you to try your own suggestion in private, but please do not talk
about it in class till I give you leave.''

If she forgot again, the teacher might say,---I think I should be
inclined to say:---``If you cannot remember not to distract the
class by talking about what is irrelevant to the business on hand, I
shall have to request you to keep outside my class-room till you
can.''

In an orderly school the teachers have time to be polite, and it is
their business to set the example of being so. In history,
especially such history as that of half-civilised countries 3000
years ago, teachers were under too much strain to cultivate either a
polite \emph{manner} of saying things, or, what is of far more
consequence, that genuine intellectual courtesy which is the
absolutely necessary condition for the development of any really
perfect mathematical system. The great Hebrew Algebra, therefore,
never became quite perfect. It was only rough hewn, so to speak; and
its manners and customs were rough too. The teachers had ways of
saying, ``Hold your tongue, or else go out of my class-room,'' which
perhaps we should now call bigoted and brutal. But what I want you
to notice is that ``Hold your tongue, or get out of my class-room,''
is not the same thing as ``My hypothesis is right, and yours ought
not to be tried anywhere.''

This latter is contrary to the essential basis of Algebra, viz., a
recognition of one's own ignorance.

The other, a rough way of saying ``Get out of my class-room,'' is
only contrary to that fine intellectual courtesy which is essential
to the \emph{perfection} of mathematical method.

\chapter[Mathematical Certainty\ldots]{Mathematical Certainty and
Reductio ad Absurdum}

It is very often said that we cannot have mathematical certainty
about anything except a few special subjects, such as number, or
quantity, or dimensions.

Mathematical certainty depends, not on the subject matter of our
investigation, but upon three conditions. The first is a constant
recognition of the limits of our own knowledge and the fact of our
own ignorance. The second is reverence for the As-Yet-Unknown. The
third is absolute fearlessness in meeting the \emph{reductio ad
absurdum}. In mathematics we are always delighted when we come to
any such conclusion as $2 + 3 = 7$. We feel that we have absolutely
cleared out of the way one among the several possible hypotheses,
and are ready to try another.

We may be still groping in the dark, but we know that one
stumbling-block has been cleared out of our path, and that we are
one step ``forrader'' on the right road. We wish to arrive at truth
about the state of our balance sheet, the number of acres in our
farm, the time it will take us to get from London to Liverpool, the
height of Snowdon, the distance of the moon, and the weight of the
sun. We have no desire to deceive ourselves upon any of these
points, and therefore we have no superstitious shrinking from the
rigid \emph{reductio ad absurdum}. On some other subjects people do
wish to be deceived. They dislike the operation of correcting the
hypothetical data which they have taken as basis. Therefore, when
they begin to see looming ahead some such ridiculous result as $2 +
3 = 7$, they shrink into themselves and try to find some process of
twisting the logic, and tinkering the equation, which will make the
answer come out a truism instead of an absurdity; and then they say,
``Our hypothetical premiss is most likely true because the
conclusion to which it brings us is obviously and indisputably
true.''

If anyone points out that there seems to be a flaw in the argument,
they say, ``You cannot expect to get mathematical certainty in this
world,'' or ``You must not push logic too far,'' or ``Everything is
more or less compromise,'' and so on.

Of course, there is no mathematical certainty to be had on those
terms. You could have no mathematical certainty about the amount you
owed your grocer if you tinkered the process of adding up his bill.
I wish to call your attention to the fact that \emph{even in this
world} there is a good deal of mathematical certainty to be had by
whosoever has endless patience, scrupulous accuracy in stating his
own ignorance, reverence for the As-Yet-Unknown, and perfect
fearlessness in meeting the \emph{reductio ad absurdum}.

\chapter{The First Hebrew Algebra}

The first Hebrew algebra is called Mosaism, from the name of Moses
the Liberator, who was its great Incarnation, or Singular Solution.
It ought hardly to be called an algebra: it is the master-key of all
algebras, the great central director for all who wish to learn how
to get into right relations to the unknown, so that they can make
algebras for themselves. Its great keynotes are these:---

When you do not know something, and wish to know it, state that you
do not know it, and keep that fact well in front of you.

When you make a provisional hypothesis, state that it is so, and
keep that fact well in front of you.

While you are trying out that provisional hypothesis, do not allow
yourself to think, or other people to talk to you, about any other
hypothesis.

Always remember that the use of algebra is to \emph{free people from
bondage}. For instance, in the case of number: Children do their
numeration, their ``carrying,'' in tens, because primitive man had
nothing to do sums with but his ten fingers.

Many children grow superstitious, and think that you cannot carry
except in tens; or that it is wrong to carry in anything but tens.
The use of algebra is to free them from bondage to all this
superstitious nonsense, and help them to see that the numbers would
come just as right if we carried in eights or twelves or twenties.
It is a little difficult to do this at first, because we are not
accustomed to it; but algebra helps to get over our stiffness and
set habits and to do numeration on any basis that suits the matter
we are dealing with.

Of course, we have to be careful not to mix two numerations. If we
are working a sum in tens, we must go on working in tens to the end
of that sum.

Never let yourself get fixed ideas that numbers (or anything else
that you are working at) will not come right unless your sum is set
or shaped in a particular way. Have a way in which you usually do a
particular kind of sum, but do not let it haunt you.

You may some day become a teacher. If ever you are teaching a class
how to set down a sum or an equation, say ``This is my way,'' or
``This is the way which I think you will find most convenient,'' or
``This is the way in which the Government Inspector requires you to
do the sums at present, and therefore you must learn it.'' But do
not take in vain the names of great unseen powers to back up either
your own limitations, or your own authority, or the Inspector's
authority. Never say, or imply, ``Arithmetic requires you to do
this; your sum will come wrong if you do it differently.'' Remember
that arithmetic requires nothing from you except absolute honesty
and patient work. You get no blessing from the Unseen Powers of
Number by slipshod statements used to make your own path easy.

Be very accurate and plodding during your hours of work, but take
care not to go on too long at a time doing mere drudgery. At certain
times give yourself a full stretch of body and mind by going to the
boundless fairyland of your subject. Think how the great
mathematicians can weigh the earth and measure the stars, and reveal
the laws of the universe; and tell yourself that it is all one
science, and that you are one of the servants of it, quite as much
as ever Pythagoras or Newton were.

Never be satisfied with being up-to-date. Think, in your slack time,
of how people before you did things. While you are at school my
little book, \emph{Logic of Arithmetic}, will help you to find out
many things about your ancestors which may amuse and interest you;
but, as soon as you leave school and choose your own reading, take
care to read up the histories of the struggles and difficulties of
the people who formerly dealt with your own subject (whatever that
may be).

If you find the whole of the data too complicated to deal with, and
judge that it is necessary to eliminate one or more of them, in
order to reduce your material within the compass of your own power
to manage, do it as a \emph{provisional} necessity. Take care to
register the fact that you have done so, and to arrange your mind,
from the first, on the understanding that the eliminated data will
have to come back. Forget them during the working out of your
experimental equation; but never give way to the feeling that they
are got rid of and done with.

Be very careful not to disturb other people's relationships to each
other. For instance, if a teacher is explaining something to another
pupil, never speak till she has done. Beware of the sentimental
craving to be ``in it.'' Any studying-group profits by right working
relations being set up between any two members; and ultimately each
member profits. The whole group suffers from any distraction between
any two. Therefore listen and learn what you can; but never disturb
or distract.%
\footnote{D.\ Marks bases the Seventh Commandment on the
desirability of not distracting existing relations.}

Take care not to become a parasite; do not lazily appropriate the
results of other people's labour, but learn and labour truly to get
your own living. Take care that everything you possess, whether
physical, mental, or spiritual, shall be the result of your own toil
as well as other people's; and remember that you are bound to pay,
in some shape or way, everyone who helps you.

Do not make things easy for yourself by speaking or thinking of data
as if they were different from what they are; and do not go off from
facing data as they are, to amuse your imagination by wishing they
were different from what they are. Such wishing is pure waste of
nerve force, weakens your intellectual power, and gets you into
habits of mental confusion.

When the time comes to stop grind-work, there is no better rest than
amusing your imagination by thinking of non-existent possibilities;
but do it on a free, generous scale. Give yourself a perfectly free
rein in the company of the Infinite. During such exercise of the
imagination, remember that you are in the company of the Infinite,
and are not dealing with, or tinkering at, the problem on your
paper.

Keep always at hand, clearly written out, a good standard selection
of the most important formul\ae{}---Arithmetical, Algebraic,
Geometric, and Trigonometrical, and accustom yourself to test your
results by referring to it.

These are the main laws of mathematical self-guidance. Once upon a
time ``Moses'' projected them on to the magic-lantern screen of
legislation. In that form they are known as the Ten Commandments;
or, to change the metaphors, we might call the Ten Commandments the
outer skin of the mathematical body.

A great many people seem to suppose that, though everyone ought to
keep the Ten Commandments, it does not matter what happens to one's
mind. Just so, there are people who live unhealthy lives, and think
they can make all right by putting cosmetics on their skin. But I
hope you have learned in the hygiene class how stupid and futile all
that is. The way to have a healthy skin is to grow it, by leading a
hygienic life on a moderate allowance of pure wholesome food, and
taking a proper amount of exercise in pure fresh air. People who do
that with their minds grow the Ten Commandments naturally, just as
Moses grew them. The world has been trying the other plan---bad food
and air inside, and cosmetics outside---for at least 4000 years; and
not much seems to have come of it yet. The Ten Commandments have not
yet succeeded in getting themselves kept. Perhaps that is why some
schoolmasters and mistresses think they would like to try the other
plan now. Still, it is very good to have a normal model of what a
healthy human being ought to look like outside. It is good to have a
standard for reference. Therefore do not get too much immersed in
the mere details of your own problems. Learn the Ten Commandments
and a few other old standard formularies by heart, and repeat them
every now and then. And say to yourself, ``If I really am doing my
algebra quite rightly, \emph{this} (the standard formularies) is how
I shall think and feel and wish. I shall wish to behave thus, not
because anybody ordered me to do so, but from sheer liking and sense
of the general fitness of things.''

\chapter{How to Choose Our Hypotheses}

The faculties by means of which we get our positive data are called
the senses (sight, hearing, etc.).

The faculty by means of which we get our hypothetical data is called
the Imagination.

Some persons are prone to warn young people against what they call
an excessive exercise of the imagination. Of course, to say that
``excessive'' anything is too much is a mere truism, but nobody
knows yet what is the proper amount of use for the imagination. What
we do know is that there is a good deal of excessive mis-use of the
imagination, by which I mean that there is a frightful amount of
using it contrary to the laws of its normal action. A kind of use of
it, such as, when we find a child doing it with its eyes, we say,
``Do not learn the habit of squinting''; or if it does the analogous
thing with its legs, we say, ``Go and run about, or do some
gymnastics; do not stand there lolloping crooked against the wall.''

Squinting and lolloping crooked are things that it is best to avoid
doing much of with any part of one's self.

Moreover, it is bad to spend too many hours over either a microscope
or a telescope, or in gazing fixedly at some one-distance range. The
eyes need change of focus. So does the imagination.

There has been in modern Europe a shocking riot in mis-use of the
imagination. The remedy is to learn to use it. But the same kind of
people who would like to bandage a child's eyes lest it should learn
to squint, like to bandage the imagination lest it should wear
itself out by squinting.

In a school which professes to be conducted on hygienic principles,
we have nothing to do with that sort of pessimistic quackery. We use
the imagination as freely as the hands and eyes.

But when we come to the end of our arithmetic we do not content
ourselves with guesses; we proceed to algebra--that is to say, to
dealing logically with the fact of our own ignorance. One of the
data that we do know is that all great nerve-centres affect each
other. Mis-use of any one tends more or less to produce distorted
action in the others. And, quite apart from that consideration, any
energetic and continued action of one tends more or less to suppress
the action of the others, for the time being, by drawing the blood
from the organs which are the seat of them; and then, when normal
circulation is restored, to produce for a time an unusual
sensitiveness in the others. There is nothing abnormal or wrong in
this, provided that we recognise the fact, and, as I said, are
careful to deal logically with the fact of our own ignorance
whenever anything happens either to our eyes or to our imagination
which we do not at the moment quite understand.

If you ever arrive at using your imagination strongly and rightly in
the construction of any sort of algebra, you may find that it
affects to some extent your sense-organs. It certainly will affect
them more or less whether you know it or not. What I mean is that it
may affect them in a way that forces you to be aware of the fact. If
ever this should happen, take it quite naturally; and as long as you
are too young to understand how it happens, just say to yourself,
``This is $x$, one of the things that I do not know, and perhaps
shall know some day if I go on quietly acting in accordance with
strict logic, and remembering my own ignorance.''

The ancient Hebrews used their imaginations very freely, and
sometimes really very logically. And sometimes the free use of the
imagination produced sensations in the eyes and ears as if of seeing
and hearing. They considered this quite natural, as it really was.
Many great mathematicians in modern Europe have had these
sensations.

The Hebrews called these sensations by a Hebrew word which is
translated by the English word ``angel,'' from the Greek
``angelos,'' a messenger. The Hebrews were quite right. The
sensations are messengers from the Great Unknown. They bring no
information about outside facts. No angel tells you how many petals
there are in a buttercup; if you want to know that, you are supposed
to ask the buttercup itself. No angel tells you the price of sugar;
you ought to ask your grocer. No angel tells you how to invest your
money; you ought to ask your banker or your lawyer. There are people
foolish enough to ask angels about investments, or about which horse
will win a race; which is just as foolish as asking your banker in
town how many blossoms there are on the rose tree in your country
garden. It is not his business, and if he made a guess it would most
likely turn out a wrong one. All that sort of thing is quackery and
superstition.

But the angels do bring us very reliable information from a vast
region of valuable truth about which most of us know very little as
yet. They guide us how to frame our \emph{next provisional working
hypothesis}, how to choose the particular hypothesis which at our
present stage of knowledge and development will be most illuminating
for us. Some of the angels come during sleep; we call them dreams.
Dreams sometimes suggest the best working hypothesis to experiment
on next. More often they warn us against thinking upon some
hypothetical basis which for the present will not suit us.

And here comes in the value of such formul\ae{} as the Ten
Commandments. They are the laws of the \emph{normal} working of the
brain machinery.

The angel (or imaginary messenger) suggests to you the one among
possible working hypotheses on which your brain will most readily
work. Now the formularies of which I spoke give you the laws of
healthy brain action. Therefore, if the angel suggests something
contrary to the registered formulas, he is suggesting the hypothesis
which you ought carefully to avoid thinking out or using at that
time. It is of all paths towards disease the one which will lead
you, in your present condition, most rapidly towards disease. But if
the imaginary angel suggests nothing contrary to the formularies,
then the image or idea which he suggests is likely to be one on
which your mind for the time being can work safely, and \emph{the}
one along which it can work most easily and profitably.

When your imagination is acting strongly in providing you with
working hypotheses, there are a few little precautions which you
ought to observe.

Do not at such times take either very rapid or very much prolonged
physical exercise.

Be rather particular not to eat anything either indigestible or
highly flavoured.

Even if you were in the habit of taking any kind of alcoholic
stimulant (which, while you are young, I hope you will not do),
avoid it during the process of framing hypotheses. Be extra careful,
at such times, to keep up any routine exercises of slack muscles and
slow breathing which you find suit you.

Take a little extra care, at such times, not to catch cold. You are
rather less liable than usual to take cold at such times; but, on
the other hand, you are less conscious than usual of ordinary
physical sensations, and may be very cold without knowing it. A
chill may settle locally, and produce permanent mischief.

Above all, be very careful, while the imaginative fit is on, to
avoid letting the subject as to which your imagination is stirred
become the object of either fun, vanity, or gossip. The vision which
you see may quite harmlessly and legitimately become a source of fun
to yourself and your friends at some future time, but take care
never to gossip or joke about it until it has passed from the
condition of imaginative vision to that of working hypothesis. But
the most important precaution of all is incessant reverence for the
Great Unknown, the sacred $x$: or, in other words, a constant
awareness of your own ignorance.

Remember always that Genius means conscientious, careful work on
suggestions of the imagination taken as provisional hypotheses.

To take suggestions of the Imagination as fact is Insanity. When you
hear of a man that he has unquestionable genius but is a little mad,
that means that he sometimes takes the products of his imagination
as working hypotheses, but sometimes mistakes them for facts.

All the above precautions may be summed up in one sentence: Remember
that the more active the imagination is, the less the physical and
moral instincts are on the alert; therefore, conscious precaution
should supplement instinct at such times, until self-protection has
become so fixed by habit as to become in its turn automatic and
instinctive.

If you observe these precautions you need not fear using your
imagination freely. When you hear of some brilliant imaginative
writer who has come to grief physically, mentally, or morally, after
a short and brilliant career, you will find it advantageous to try
to find out which of the precautions he has been neglecting.

In future letters I hope to point out to your notice some famous
cases of disaster due to such neglect.

\chapter{The Limits of the Teacher's Function}

One of the greatest causes of mental and moral confusion, as well as
of absolute insanity, in modern Europe, is the fact that numbers of
people plunge into the second and third great Hebrew algebras before
they rightly understand the first. Even if they are silent about
their results, this distracts their own minds, and sows the seeds of
bad habits and mental confusion in their own constitutions. Many of
these people give to the world their own wild guesses about the
second and third algebras, and that puts the rest of the world into
confusion. We are, therefore, not going to enter on the question of
the second algebra till I have provided you with the possibility of
understanding and practising the first. In the next few chapters I
hope to give you a series of stories of people who used, and
sometimes mis-used, the algebra of Moses, in order that you may see
how to work the rules strictly and how mistakes might creep in.

But, before we begin our stories, there is one principle to which I
must call your attention: it is the business of your teachers at
school to see that you acquire skill in using certain implements or
tools; it is not their business nor mine to decide what use you
shall make, when you are grown up, of the skill which you have
acquired. It is their business to see that you learn to read and to
speak properly; it is not their business to decide beforehand
whether you shall recite in public or only read to your own family
and your sick friends. It is their business to see that you know how
to sew; but not to settle whether you shall, in future, make your
own clothes or work for the poor. So it is with the tools of the
mind, such as algebra and logic. It is our business to see that you
know how to use algebraic and logical method accurately and
skilfully; it is not our business to decide whether, in the future,
you shall use your skill to deceive other people or to show them the
truth. It \emph{is} our business to see that you do not deceive
yourself, because deceiving \emph{yourself} distorts your brain and
ruins the possibility of using logical methods skilfully to arrive
at the knowledge of truths.

When you have found out a truth, then the question whether you shall
or shall not tell it to other people is a matter of conscience. You
will have to settle it alone with the Great Power which no man
knows. Self-deception, slipshod logic, and bad algebra are things
which it is the business of your elders to protect you from while
you are young, in order that you may not \emph{lose the power} of
being honest in case you wish to be so. My business is not to judge
what is good or bad conduct, but to see that you learn how to be
perfectly honest with yourself. I wish you to notice this, because
in the books of the Hebrew algebra you will sometimes find good kind
people spoken of very harshly; and some of the most dishonest and
selfish people in the world praised and spoken of as blessed. This
puzzles many good people, because they choose to fancy that the
Hebrew books are sermons about right and wrong feelings; and do not
like to recognise that they are really about the algebra of logic.

As I said before, people who really conduct their minds strictly
according to the algebra of logic are very prone to grow kindness
and honesty towards other people, without thinking about it, as a
matter of taste, of choice. They \emph{like} being kind and honest
better than being selfish and dishonest, and they become kind and
honest without thinking much about it. But honesty to other people
and honesty to yourself \emph{are} two different things, and must be
kept apart in your mind, just as, in physiology class, you keep
apart the flesh of an animal and its skin. You believe that if the
flesh is thoroughly healthy it will grow a good skin; but, while you
are studying, you do not mix up statements about the one with
guesses about the other. If we find that a man's logic was good, and
his conduct what we should call bad, we must do what a doctor would
do if he found a spot on a patient's skin which he could not account
for by anything wrong in his circulation or digestion. He ought not
to say either, ``That spot is not there,'' or, ``I suppose it is
right that spot should be there,'' nor, on the other hand, to jump
to the conclusion that that patient had been eating some
particularly unwholesome thing. He ought to register in his mind, as
one of his data, the fact of his own ignorance of how that spot came
there. I shall have to tell you in another chapter the story of one
of the most selfish and deceitful persons that ever lived, as to his
conduct towards other people, but who was said to be blessed,
apparently for no reason except that he was absolutely straight with
his logic and honest with himself.

Besides, no one who is consciously and deliberately dishonest to
serve his own selfish purposes can ever do as much harm to other
people as is done every day by men and women who have muddled their
own brains with crooked logic.

\chapter{The Use of Sewing Cards}

When you go for holidays perhaps your friends will ask you what is
the use of sewing curves on cards. I should like you to know exactly
what to say.

The use of the single sewing cards is to provide children in the
kindergarten with the means of finding out the exact nature of the
relation between one dimension and two.

There is another set of sewing cards which is made by laying two
cards side by side on the table and pasting a tape over the crack
between them. This tape forms a hinge. You can lay one card flat and
stand the other edgeways upright, and lace patterns between them
from one to the other.

The use of this part of the method is to provide girls in the higher
forms with a means of learning the relation between two dimensions
and three.

There is another set of models, the use of which is to provide
people who have left school with a means of learning the relation
between three dimensions and four.

The use of the books which are signed George Boole or Mary Everest
Boole is to provide reasonable people, who have learned the logic of
algebra conscientiously, with a means of teaching themselves the
relations between $n$ dimensions and $n + 1$ dimensions, whatever
number $n$ may be.

The above is a quite accurate account of the real Boole Method; as
much as there is any need for you to know while you are at school.

I should feel grateful to you if you will each copy it out in a
clear handwriting, and keep it by you, and take it home whenever you
go away from school for the holidays. It would be all the better if
you learned it by heart.

And now I will tell you why I am so anxious about this.

The Boole method is a conveyance which will take you safely to
wherever the Great Unknown directs you to go. Some people mistake it
for the carpet in the \emph{Arabian Nights}, which took whoever
stepped on it wherever he or she \emph{wished} to go--which is a
quite different thing. The true Boole method depends essentially on
making a right use of imaginary hypotheses. The magic carpet depends
for its efficacy on making a wrong use of imaginary hypotheses.

People get to very queer places on that carpet. I have been for
several excursions on it, so I know.

One of the places it can take you to is a town where all the front
doors open on to a street very like Regent Street; with the most
gorgeous millinery, jewellery, and fruits in shop windows; and all
the back doors open to wild country where blue roses, black tulips,
and the fattest double carnations of all colours (including green
ones) grow wild in the hedges and fields; and where all the pigs
have wings.

Another place that it can take you to is one where pigs can wallow
in all the filth they like without soiling their wings; and moths
fly into candles without singeing theirs.

The carpet will take you straight \emph{to} whatever place you wish
to go to. It is by no means warranted to take you safely back.

The advantage of Boole's method is that it \emph{is} warranted to
bring you safe down somewhere on solid earth,---not always the exact
place you started from, but a safe and clean place of some
kind---and to deposit you steady on your feet, with a compass in
your pocket which will show you a straight way home.

\chapter{The Story of a Working Hypothesis}

In an old Hebrew book there is a story of a person named Jacob,
which means the Supplanter. If you want to know why, you had better
read the story for yourself some day. It is not entirely a pretty
story, but it is very instructive. Jacob had a dream in which he saw
``angels'' coming down a ladder. It would be a very profitable
exercise of your imagination to ask yourselves why this particular
patriarch saw angels on a ladder, whereas so many other Hebrews saw
them in clouds, or flying down on wings, or mixed up with flames and
other romantic, pretty, moving things.

Jacob had another dream, and saw an angel who wrestled with him, and
apparently left him with sciatica for life; which is not surprising,
for he had been sleeping out of doors on bare ground, just when
\emph{he} had been wrestling with very serious difficulties caused
by his own dishonest tricks. At such times, as I told you before,
people had better be a little extra careful not to catch cold;
because colds caught under such conditions are rather prone to leave
unpleasant traces, which last a long time, and sometimes all one's
life.

Well, the angel who gave Jacob sciatica gave him something else: a
new name. Why did he give him a new name? Taking a new name was an
ancient ceremony which meant entering a new service. Sixty years ago
servants in Devonshire were called by their employer's name. A
gardener would have two names---his own, which he got from his
father, and his master's. I have even heard dogs called by their
master's names, for instance, Toby Smith, or Ponto Jones.

You will often notice in old books that when people were converted,
that is to say, when they either took up a new religion or turned
from bad ways to good ones, the people who persuaded them to be
converted gave them a new name, very often the teacher's own name.
Well, the angel who wrestled with Jacob appears to have converted
him. He seems to have persuaded Jacob that there are other ways of
getting on in the world and promoting the fortunes of one's children
and grandchildren besides cheating everybody, including one's own
nearest relations.

Therefore Jacob was not to be called ``the Supplanter'' any more:
his new name was to be Isra�l. Jacob's descendants are called
Hebrews, and also ``the people of Isra�l.'' Isra�l was the new name
which Jacob got when he turned from cheating to a better way of
getting on in life.

What was that better way? That is our $x$, our first unknown. What
does the word Isra�l mean? That is our $y$, our second unknown. I
may as well tell you at once that, so far as I am concerned, $y$
remains unknown. I want you to take notice that \emph{I} do not know
what the word Isra�l means. But some twenty years ago my imagination
supplied me with a working hypothesis:--Suppose Isra�l meant rhythm.

Now if I had gone telling people that \emph{Isra�l means rhythm}, I
should have been contradicted and laughed at and told that I had no
proof of what I said and was talking of what I knew nothing about;
and whoever said so would have been perfectly right. I should have
been cheating myself and getting into bad slipshod habits. What I
did was to post up inside my brain as a working hypothesis:
``\emph{Suppose} Isra�l means rhythm, what would be the consequence
of that hypothesis?'' Then I read through old books of the Hebrews,
putting in my mind the word ``rhythm'' wherever I found the word
``Isra�l,'' and ``the people of rhythm'' instead of ``the people of
Isra�l.''

In the stories that are told about Jacob and his grandfather Abraham
the angels are represented as telling the two men that if they would
obey the angels, not only they themselves would be blessed, but all
their descendants would be blessed too, and be made, at last, the
means of conferring a great blessing on all the world; Moses warned
them that, if they did not obey their own special angels, some
special trouble would come to them.

My imagination suggested to me that perhaps getting into the swing
of rhythmic beats is good for all people, but more good for the
people of Isra�l than for anybody else; and that wandering off into
irregular un-rhythmic freaks is more bad for the people of Isra�l
than for anybody else.

This, again, you will observe, is purely imaginary hypothesis. I had
not the faintest warrant for saying anything of the kind; therefore
I did not say it; but I experimented at treating my Hebrew friends
and acquaintance \emph{as if} they were natural born ministers, or
servants, of the principle of rhythmic beat; as if it was their
business to introduce respect for rhythm and an orderly arrangement
of time into the general morals of the world; and as if they would,
of course, become degraded more than other people, if they allowed
themselves to drift into being irregular and disorderly. Now you
will observe that, though all this was purely imaginary hypothesis,
it was of a harmless kind; there is nothing contrary to the ten
commandments, or to any other register of safe rules, in treating
one's Hebrew acquaintance as if one expected them to be more orderly
as to time than other people.

The registered rules allowed me to consider this a safe road; and my
imagination showed me that it was one along which I could travel
quickly; therefore I started to go along it and waited to see where
I got to. One consequence which came was that some of the people of
Isra�l began telling me that I seemed to know things about their old
books (even some old books that I had never read), which they
themselves had never observed before; I had enabled them to get at
real values for the $x$'s and $y$'s; of some of their problems.

Please notice that all this is pure imaginary hypothesis. Ancient
peoples made a hypothesis, for which they had no authority, about
angels; and I made one, for which I had no authority, about some of
those supposed angels. And, by dealing logically with these
imaginations, we got to some very real knowledge.

\chapter{Macbeth's Mistake}

The whole question of choosing one's next working hypothesis has
been fogged, owing to people's neglect of a very simple principle.
Suppose you are out bicycling in a strange place. You come to a bit
of smooth, good road, which is either flat or goes very gently down
hill; and presently curves in a nice, big, easy sweep round a bit of
wood or a cliff, so that you cannot see far along it. What you know
at once is that you can, \emph{if you choose}, get up great speed
without overmuch exertion. That is obvious, and needs no discussion.
The question you have to settle is: Shall you choose to do it?

If you have heard the whole road spoken of, in general terms, as a
nice safe one to go on, you probably do choose to make use of the
specially easy bit of the road to get up a lively spin.

But supposing that, at the beginning of the gentle slope down, you
come upon a notice board with an inscription ``Go slowly,'' or
``Dangerous to cyclists,'' I hope you would have sense enough not to
think---``What do those old fogies know about the needs of the young
generation? I have a right to go fast if I choose, and I shall have
my jolly spin in spite of them.'' Nor would you say: ``I can take
care of myself, and if I run into somebody else that is his look
out.'' If you are an experienced cyclist you would keep on your
seat, and go cautiously; if you are still a very inexperienced one,
it would be wise to get off your cycle, and not mount again till you
had come to the curve, and gone round it, and seen what is beyond.

The notice board is not an actual prohibition to go along the
``King's highway'' if you choose. The people who put up the board
have no authority over you. But your own instincts of
self-preservation, and I hope also your instinct of loyalty and good
comradeship with the possible other cyclist who may be at the bottom
of the hill, would suggest to you not to throw away the guardianship
of a caution from those who know more than you do about the road.

Having given you this general indication of the principle which I am
trying to explain, we will go back to the question of an imaginary
working hypothesis.

My imagination, as I told you, showed me that my mind would travel
quickly and easily along the road opened up by supposing that Isra�l
means Rhythm. Looking back in my memory, I could not find the
smallest indication that anybody had either come to grief himself or
offended any Hebrew person by behaving as if the people of Isra�l
were the People of Rhythm; and there is nothing in the Ten
Commandments to suggest that there is any harm in doing so. So I
started off on a glorious, easy, rapid spin; and arrived, without
any mishap, at several very interesting bits of scenery.

Now let us take the case of the old Scotch legend of Macbeth, as
told by Shakespeare.

Macbeth and his wife appear to have been, at first, very
well-intentioned, good people, as human beings go; better than most
people; and enormously better than Jacob, or his mother, or his
uncle, or most of the people belonging to him. Macbeth was a
brilliant and successful soldier; his imagination suggested to him
that he had it in him to rise rapidly to fortune and power. He might
become Thane of Cawdor, and some day even King of Scotland. His
imagination was so vivid that he pictured three old women going
through some heathen incantation and predicting to him that he would
be Thane of Cawdor and King. Here was a road open, along which it
was quite sure that his mind would travel easily if he would let it
do so. The question was: Should he let it go along that road? Now
there were living at the time a Thane of Cawdor and a King of
Scotland. While they lived, he could not be either. The commandments
say, ``Thou shalt not covet thy neighbour's goods.'' Here was a
danger signal. If Macheth had known as much as Shakespeare knew
about the art of sound thinking, he would immediately have said to
himself, ``\,`Cawdor' and `King' are the roads that I had better not
travel along just now, for fear the wheels of my mind should get too
much way on, and carry me into danger.'' But Macbeth had either not
learnt algebra at school, or, if he had, he had only crammed it up
for examination out of a textbook, and not learned it as the Science
of the \emph{Laws of Thought}.

Another day his imagination showed him a dagger. A dagger is a thing
to kill people with. As a soldier, he had probably used a real one
in war. But, if he had had any proper nerve training, he would have
known that when his imagination was so vivid that he did not, for
the moment, know an imaginary dagger from a real one, he ought
immediately to ``go slack''; to lie down and think about the moors
or the sky, or about anything or anybody that was not connected with
doing anything in particular, with planning anything, with taking
any resolution, and especially with breaking any of the Ten
Commandments. He had already told his wife about the three old
women. If she had been a sensible woman, she would have told him
that she wanted to go away from home; and got him to take her right
away for a few weeks; and kept him busy and amused in thinking of
other things; till he left off seeing things that were not there.
But neither Macbeth nor his wife knew as much as Shakespeare did
about the value of danger signals and the conditions for making a
safe working hypothesis.

You had better read the story of Macbeth and see for yourselves what
they did do.

Next to the old Hebrew books, Shakespeare is the best road map that
I know of for people who wish to travel safely about the country of
the imagination.

\chapter{Jacob's Ladder}

In Chapter X.\ I set you children a question:---Why did Jacob's
angels come down a ladder, whereas other Hebrews saw angels mixed up
with romantic pretty things such as wings and clouds?

I hope some of you have made a guess before now; but some are not
good at guessing. I will tell you what may help you to find out.

If a bird wants to go up and down from the roof to the garden, it
trusts to its wings. A man has to use a ladder:
step,---step,---step.

If a bird is not fully fledged or has a broken wing, it has to find
something more or less like a ladder; and go up and down bit by bit:
hop,---hop,---hop.

If an artist wishes to draw a parabola, he does it freehand, that is
to say, he just draws the curve He does not take all the trouble
which Mrs Somervell's book makes little children take, of getting
the curve step by step by the method of Finite Differences.

Jacob wished to be rich. Some angel, but a very bad one, inspired
him with an idea of getting rich in one big sweep, by cheating his
father and brother. By wanting to do things in that sort of quick,
easy way, when he did not yet know how to do things both quickly and
rightly, he got into terrible trouble and had to leave his country.

Now I suppose that the angels who converted him meant to say
something like this: ``It is all very well for good, holy,
God-fearing men like your father and grandfather to go where they
are taken by angels who can move about on wings; but you are at
present a stupid, clumsy person; your wings have not grown yet, or
you have broken them by being covetous. We are going to show you how
\emph{you} should go about: step,---step,---step. Have patience, and
take pains; and don't go about on magic carpets.''

\chapter{The Great $x$ of the World}

A great question which people like to quarrel about is:---Who or
What made things be as they are? As soon as people grew clever
enough to think about anything except scrambling for food and taking
care of their own babies, they began quarrelling about Who or What
made things be. Nobody knew anything about it; and most people had a
great deal to say about it. Moses saw that there was no hope of
getting a country orderly while all this confusion was going on; so
he said to the Hebrews, ``I must not allow all this confusion to go
on among a people that I am made responsible for. None of us have
ever seen the Maker of things. We can see the things growing, but
not the force that makes them. \emph{That} is our X; our Unknown. We
are going to begin by stating that we don't know. We are going to
call the Maker of things `I Am,' or `That which is, whatever it is';
and we are going to make two hypotheses to start with. We are going
to try thinking of `I Am' as Unity; one, and not several or a
fraction. We will also try thinking of `I Am' as No-Thing,---we are
not going to suppose at present that any particular kind of thing
made the rest; we will suppose that `I Am' is not a thing. When we
find that any particular proceeding or behaviour destroys men, or
makes them too sickly or weak or stupid or quarrelsome to manage
other creatures and keep the upper hand of the world, we will say,
for short, that `I Am' does not like or does not intend the people
of Isra�l to go on with that kind of proceeding or behaviour.

``Now these two hypotheses are as much as we can deal with for the
present. Anybody who wants to think out other hypotheses than those
will have to think to himself, or go out of the country that I am to
manage.

``Now we will arrange all the facts that we know round the statement
of our own ignorance; and then try our hypotheses on them.

``We know that eating the flesh of certain uncleanly animals gives
people certain diseases; we will say, for short, `I Am' does not
intend the Hebrews to eat the flesh of those animals. We know that
if people are dirty in their habits and careless in preparing their
food and in washing their hands before they touch food, they get
fevers; we will suppose that `I Am' does not intend the people of
Isra�l to be dirty in their habits. We know that if people burn
things the smoke of which makes them drunk and silly, they manage
their affairs badly, and make mistakes, and do not grow their crops
properly, and are not ready to fight when enemies attack them. The
people in neighbouring countries say that the Maker of things likes
or dislikes to smell the smoke of these drugs; they know no more
than we do what \emph{He} likes to smell, but we are going to
suppose that `I Am' does not like \emph{us} to smell them.''

The Hebrews never found out what ``I Am'' is; but those who stuck
loyally by the hypotheses of Moses, and refused to be distracted
from the matter in hand, or to talk about anything except the
experiment which they were trying, found out several things that
were very useful to them. For instance, about weather and the
electricity of the atmosphere, and how to take care of their health,
and how to use their imagination to supply them with working
hypotheses for a variety of sciences, and how to use their dreams to
show them where they had been making mistakes and spoiling their
brains. Whereas the people who would insist on shouting and arguing
and quarrelling about things which were only wild guesses got on
very slowly with learning Science.

\chapter{Go Out of My Class-Room}

A story is told of one of the orderly pupils of Mosaism who got to
know a good deal about weather and electricity; and at last he got
out of patience with the people who wanted to shout and argue. And
he said to them: ``What is the good of all this arguing backwards
and forwards about things that we do not know and cannot settle? Let
us try a fair experiment. You go on shouting and doing whatever
\emph{you} think the Unseen Powers like; and I will do what \emph{I}
think will get them to do what \emph{I} like. And let us agree that
whichever of us can draw a spark out of a thundercloud shall be
considered to know most about how to come to an understanding with
`I Am.'\,''

So the other people shouted and jumped about, and cut themselves
with knives; because they had taken it into their heads to imagine
that the Maker of things liked to see that kind of behaviour.

Why they thought so I cannot conceive. But there's no end to the
rubbish that people get to think when they argue about what X is,
instead of trying hypotheses in an orderly manner.

The Unknown Powers let them shout all day long; and then Elijah got
a spark out of a thundercloud.

The same sort of thing happened again about a hundred and fifty
years ago. Various sorts of priests were shouting and arguing about
what ``I Am'' wished people to believe and to think; and then
Benjamin Franklin and his friends, who had not been mixing up with
the argument or making wild guesses, but quietly experimenting and
dealing logically with the fact of their own ignorance, sent up a
kite into a thundercloud, and got a spark down; and the consequence
of that is that all kinds of people say, ``What a wonderful man
Benjamin Franklin was!'' and all sorts of people are able to ride
about in electric trams.

But the curious part of the matter is that many people use electric
trams to go to meetings, on purpose to shout and argue and make wild
guesses about things they know nothing about!

However, what they choose to do is not our business. You are living
in an orderly school; and of course you do not argue about things
you know nothing about. Let us go back to our Hebrew electrician.

He had shown the people of Isra�l what comes of sticking peaceably
to one's working hypothesis. If he had been thoroughly logical he
would have gone on sticking to it. He would have said to the people
of Isra�l, ``Now you see that I can teach you electricity; this land
is going to be my class-room; make those shouting people hold their
tongues, or else go away; so that we can go on with our lessons in
peace. When they want to learn electricity properly, they can come
back.'' But he was in too great a hurry to make a complete and final
settlement. A good teacher sends a noisy, troublesome pupil out of
his class-room for the time, but does not expel her from the school
merely for being troublesome. The shouting people were among the
facts which ``I Am'' put before Elijah to deal with. He found it
necessary to eliminate them in order to reduce his data within the
compass of his power to manage, but he should have done it as a
provisional necessity. He should have arranged his mind on the
understanding that the eliminated data would have to come back.

Instead of that he used his power and science to kill them; and gave
way to the feeling that they were got rid of and done with.

And then his mind began to go wrong. He lost his nerve. He began to
talk nonsense about things \emph{he} knew nothing about, and led a
great many people into mistakes.

\chapter{$\sqrt{-1}$}

When you come to quadratic equations you will be confronted with an
entity (or non-entity) whose name is written this way---$\sqrt{-1}$,
and pronounced "square root of minus one." Many people let this
nonentity persuade them to foolish courses. A story is told of a man
at Cambridge who was expected to be Senior Wrangler; but he got
thinking \emph{about} the square root of minus one as if it were a
reality, till he lost his sleep and dreamed that \emph{he} was the
square root of minus one and could not extract himself; and he
became so ill that he could not go to his examination at all.
Angels, and square roots of negative quantities, and the other
things that have no existence in three dimensions, do not come to us
to gossip about themselves; or the place they came from; or where
they are going to; or where we are going to in the far future. They
are messengers from the As-Yet-Unknown; and come to tell us where we
are to go next; and the shortest road to get there; and where we
ought not to go just at present. When square root of minus one comes
to you, behave reasonably about him. Treat him logically, exactly as
if he were six or nine; only always remember to keep well in front
of you \emph{the fact of your own ignorance}. You may never find out
any more about him than you know now; but if you treat him sensibly
he will tell you plenty of truths about your $x$'s and $y$'s, and
other unknown things.

Please don't suppose that I have always behaved sensibly to angels.
I have often made serious mistakes in dealing with them. I have
acted in haste and have had plenty of reason to repent at leisure.
But one thing they have taught me is that we need never be
\emph{afraid} of angels, whether white or black, as long as we keep
the laws of logic. Another thing they have shown me is that angels
never really gossip. They have often pretended to gossip to me; but
I have found out afterwards that they have been talking clever
nonsense in order to test me and prove me; so that I might see in
what an illogical state of mind I have met them. Angels leave real
gossip to old women who have done their life's work and have time to
sit in the chimney corner and tell tales about their past
experiences to their child friends.

\chapter{Infinity}

You remember the angel who looks like this, $\sqrt{-1}$. Now I am
going to introduce you to another angel. It is called ``Infinity.''
When you come to it, remember what I told you before---Angels are
messengers from the great world of the ``As-Yet-Unknown.'' They
never gossip about their private affairs, or those of other angels.
They come to tell you either about what you are to do next, or about
something you had better not do next; and if you ask them
impertinent questions about things that do not concern you for the
time being, they will give you headaches and make your head spin:
just to teach you to mind your own business. This particular angel
always comes with a message about a broken link or a loosened chain.
It comes, when an hypothesis has been fully worked out, to tell you
that you are now free from the bonds of that hypothesis and at
liberty to start experimenting on a fresh one. But its message is
never: ``You have got out of that particular house of bondage and
therefore you may, for all the rest of your life, run riot, and eat,
and drink, and do, whatever you please.'' Its message always is:
``You have outgrown that master; now you may take a holiday and have
a fling before you go into a higher class; but, just because you are
set free, look out for danger traps; and mind your Ten
Commandments.''

You will understand Infinity's messages better if you will read
carefully what is written about it in Chapter XV. of ``The Logic of
Arithmetic.'' It brought the answer to the question: ``How many
children could pass through a school-room without the apples all
being eaten up, supposing that none of the children ate any?''

Let us go over that ground again. Suppose there is a cake on the
table. How many children can go through the room without the cake
being all eaten up?

Well, that depends on two things: the size of the cake, and the
share which each child eats. If the cake weighs two pounds, and each
child eats two ounces, it will be all eaten up when sixteen children
have gone through the room. If the cake weighs only one pound, it
will be eaten up when eight children have gone through the room. But
if each child eats only one ounce, then again sixteen children will
have to go through the room before the cake is eaten up, and so on.
Many questions could be asked, all depending on the size of the cake
and the size of each child's share.

All this time you are tied to an hypothesis that the children eat
cake (more or less).

But now suppose we are freed from that hypothesis. Suppose no cake
is given to the children. How many can pass through the room before
it is all eaten up?

The answer to that is: ``An infinite number.'' Infinity does not
mean any particular number, or a very large number. It means a
loosened chain, a discarded hypothesis, escape from the rule we were
working under. Something else, not the size of the cake, determines
the number of children. Infinity does not mean that there are enough
children in the world now to go on passing through the room
\emph{for ever}, but that the number of children who pass through
the room, now that the share of each child is 0 (zero), will have to
be determined by the number of children that there are in the
school, or the parish, or wherever it is that the children are
supposed to come from; \emph{and not by the size of the cake}. The
size of the cake has no longer anything to do with answering the
question: ``How many children can pass through the room before the
cake is all eaten?''

\chapter{From Bondage to Freedom}

Moses had said, from the first, that the people of Isra�l would have
to think of ``I Am'' as the deliverer from bondage; but they were
not, at the time when he said it, advanced enough in their algebra
to understand that idea properly. So he gave them, as an hypothesis
to work on for the time being, that ``I Am'' did not like the
\emph{people of Isra�l} to eat and drink and smell unwholesome
things. He wished to make them attend to their own affairs, and
think as little as possible about what was done and thought outside
of their own land.

But, after the time of Elijah, there came a change. A higher kind of
algebra came into use. Its incarnation was called Isaiah.

The imagination of the Hebrews broke loose from the hypothesis that
``I Am'' had wishes and likes about the people of Isra�l different
from what was right for all the rest of the world.

When that hypothesis was taken away, the imagination of such people
as Isaiah took wings and flew to---well---we do not know where, but
we call it Infinity. We know nothing about Infinity; except that it
comes when a chain is loosened.

If you want to understand what it was that happened to Isaiah, and
what Infinity means in algebra, this is how you can find out. Get a
bowl and dip up some of the water out of a barrel in which a gnat
has laid her eggs. Little wigglers are born from those eggs. If you
watch them you will see that they swim in different positions, some
with their tails uppermost, some with their heads uppermost. There
may also be some worms, who do not swim much, but wriggle about at
the bottom of the bowl. Perhaps if we could hear them talking we
should hear them quarrelling about which was the right position.
Some of them might be disputing about what would happen to them in
the future. They might quarrel till the end of the world, and know
no more about it at the end than at the beginning. They are all tied
by the same hypothesis:---that everybody lives under water. It is a
very good working hypothesis for them; for if one of them got out of
the water it would die. If they knew algebra properly, they would
understand that water is only their present working hypothesis, and
that it is quite possible there may be people who live out of it.
But it is not sure that they do know enough algebra to be
\emph{aware of their own ignorance}.

If you watch them carefully, you will some day see a wiggler come
out of the water. He has got wings. The water-hypothesis no longer
concerns him. Some link in the chain that bound him down to water
has opened; he is set free; Infinity has come to him.

That is what happened to Isaiah when he got out of the kind of
Mosaism by which such people as Joshua and Samuel were tied down.
That is what will happen to you (if you learn your algebra properly)
when you are no longer tied down to $a$, $b$, $c$, and $\sqrt{-1}$,
as the values of $x$; and learn to see that the answer to a problem
may sometimes be
\begin{equation*}
X = Infinity.
\end{equation*}

Please notice that if a winged gnat fell back into the water he
would die. You will find this a good working rule:---Whenever
anything comes near your imagination which calls itself either
``Infinity'' or ``The Liberation from Bondage,'' go slack for a few
minutes; say over the Ten Commandments; and make a mind-picture of
the gnat-grub in the water. Tell yourself that his best chance of
growing strong wings and being able to fly, when Infinity comes and
calls him to go up higher, is to stay in the water till the wings
have grown strong and work out the water-hypothesis to its logical
conclusion.

Then make another mind-picture:---The gnat who has got wings, and
\emph{therefore must not try to amuse himself in the water}.

Please observe:---There is nothing in this rule contrary to any
commandment. Moreover, there is nothing slavish or degrading in it;
nothing in the least like giving up your own liberty, or hampering
your own initiative, or being a slave to past ages; nothing which
prevents your being up to date and fit for the generation to which
you belong.

You are not asked to have any opinions about it; or to think that it
is a duty in itself; or to think that you are better than other
people because you do it, or that every one is wrong who does not do
it. If you do it, it will be for no reason that you know of, except
that an old woman who has been trying to amuse you asks you to do it
as a token of friendly feeling towards her.

\chapter{Appendix}

The essential element of Algebra:---the habitual registration of the
exact limits of one's knowledge, the incessant calling into
consciousness of the fact of one's own ignorance, is the element
which Boole's would-be interpreters have left out of his method. It
is also the element which modern Theosophy omits in its
interpretation of ancient Oriental Mind Science.

Men who wish to exploit other men fear nothing in logic or science
except this element. They fear nothing in earth, heaven, or hell, so
much as a public accustomed to realise exactly \emph{how much has
been proved, and where its own ignorance begins}. Exploiters fear
this about equally, whether they call themselves priests,
schoolmasters, college dons, political leaders, or organisers of
syndicates and trusts.

As long as general readers can be kept from the habit of registering
at every step the fact of their own ignorance and the limits of
their own knowledge, a clever charlatan can deceive them about
anything he pleases:---``from pitch-and-toss to manslaughter''; from
Zero to Infinity; from the contents of a meat tin to the contents of
an engineer's report; from the interpretation of a bill before
Parliament to the interpretation of Isaiah.

Once get any fair proportion of the public into the steady habit of
algebraising ignorance, and you will have done much towards reducing
all kinds of parasitic creatures to the alternative of starvation,
suicide, or earning their own living by rendering some kind of real
service to the organism which supports them.

\markright{ADVERTISEMENT.}
\begin{center}
\fbox{\parbox{11cm}{
\begin{center}
\Huge\textbf{Logic Taught} \\
\textbf{by Love} \\
\bigskip\Large\textbf{RHYTHM IN NATURE AND} \\
\textbf{IN EDUCATION} \\
\smallskip \normalsize A set of articles chiefly on the light thrown
on each other \\
by Jewish Ritual and Modern Science \\
\smallskip \textbf{By} \\
\Large\textbf{Mary Everest Boole} \\
\smallskip \normalsize Crown 8vo, Cloth, 3s.\ 6d.\ net. \\
\smallskip \textbf{LIST OF CHAPTERS} \\
\end{center}
\footnotesize\begin{itemize}
  \item[1.] In the Beginning was the Logos.
  \item[2.] The Natural Symbols of Pulsation.
  \item[3.] Geometric Symbols of Progress by Pulsation.
  \item[4.] The Sabbath of Renewal.
  \item[5.] The Recovery of a Lost Instrument.
  \item[6.] Babbage on Miracle.
  \item[7.] Gratry on Logic.
  \item[8.] Gratry on Study.
  \item[9.] Boole and the Laws of Thought.
  \item[10.] Singular Solutions.
  \item[11.] Algebraizers.
  \item[12.] Degenerations towards Lunacy and Crime.
  \item[13.] The Redemption of Evil.
  \item[14.] The Science of Prophecy.
  \item[15.] Why the Prophet should be Lonely.
  \item[16.] Reform, False and True.
  \item[17.] Critique and Criticasters.
  \item[18.] The Sabbath of Freedom.
  \item[19.] The Art of Education.
  \item[20.] Trinity Myths.
  \item[21.] Study of Antagonistic Thinkers.
  \item[22.] Our Relation to the Sacred Tribe.
  \item[23.] Progress, False and True.
  \item[24.] The Messianic Kingdom.
  \item[25.] An Aryan Seeress to a Hebrew Prophet.
  \item[ ] Appendix I.
  \item[ ] Appendix II.
\end{itemize} \normalsize
\begin{center}
\rule{10cm}{1pt}
\textbf{London: C.W.\ DANIEL, 11 Cursitor Street, E.C.}
\end{center}}}
\end{center}

\newpage
\begin{center}
\fbox{\parbox{11cm}{
\begin{center}
\Huge\textbf{The Message of} \\
\textbf{Psychic Science to} \\
\textbf{The World} \\
\smallskip \normalsize \textbf{By} \\
\Large\textbf{Mary Everest Boole} \\
\smallskip \normalsize Crown 8vo, Cloth, \textbf{3s.\ 6d.\ } net. \\
\smallskip \textbf{LIST OF CHAPTERS} \\
\end{center}
\footnotesize\begin{itemize}
  \item[1.] The Forces of Nature.
  \item[2.] On Development, and on Infantile Fever as a Crisis of Development.
  \item[3.] On Mental Hygiene in Sickness.
  \item[4.] On the Respective Claims of Science and Theology.
  \item[5.] Thought Transference.
  \item[6.] On Hom\oe{}opathy.
  \item[7.] Conclusion.
  \item[ ] \quad APPENDIX:---
  \item[ ] On Phrenology.
  \item[ ] Notes.
\end{itemize}
\normalsize
\begin{center}
\rule{10cm}{1pt}
\textbf{London: C.W.\ DANIEL, 11 Cursitor Street, E.C.}
\end{center}}}
\end{center}

\newpage
\begin{center}
\fbox{\parbox{11cm}{
\begin{center}
\Huge\textbf{Mistletoe and Olive} \\
\normalsize An Introduction for Children to the Life of
   Revelation \\
\Large\textbf{By Mary Everest Boole} \\
\smallskip \normalsize Royal 16 mo. Cloth, \textbf{1s.\ 6d.\ } net. \\
\smallskip \textbf{LIST OF CHAPTERS} \\
\end{center}
\footnotesize\begin{itemize}
  \item[1.] Greeting the Rainbow.
  \item[2.] God hath not left Himself without a Witness.
  \item[3.] Out of Egypt have I called my Son.
  \item[4.] Holding up the Leader's Hands.
  \item[5.] Greeting the Darkness.
  \item[6.] Blind Guides.
  \item[7.] Hard Lessons made Easy.
  \item[8.] The Cutting of the Mistletoe.
  \item[9.] Genius comes by a Minus.
  \item[10.] The Rainbow at Sea, or the Magician's Confession.
\end{itemize}
\begin{center}
\rule{10cm}{1pt}
\smallskip \Huge\textbf{Miss Eduction} \\
\textbf{and Her Garden} \\
\smallskip \normalsize A Short Summary of the Educational \\
Blunders of half a century \\
\Large\textbf{By Mary Everest Boole} \\
\smallskip \normalsize Royal 16 mo. Cloth, \textbf{6d.\ } net. \\
\rule{10cm}{1pt}
\textbf{London: C.W.\ DANIEL, 11 Cursitor Street, E.C.}
\end{center}}}
\end{center}

\newpage
\begin{center}
\fbox{\parbox{11cm}{
\begin{center}
\Huge\textbf{Mathematical} \\
\textbf{\quad Psychology of} \\
\textbf{Gratry and Boole} \\
\smallskip \normalsize \textbf{for MEDICAL STUDENTS} \\
\smallskip Dedicated, by permission to Dr. H. \textsc{Maudesley} \\
as a contribution to the science of brain, showing \\
the light thrown on the natur of the human \\
brain by the evolution of mathematical process. \\
\smallskip \textbf{By}
\Large\textbf{Mary Everest Boole} \\
\smallskip \normalsize \textbf{Crown 8vo. Cloth, 3s.\ 6d.\ net.} \\
\smallskip LIST OF CHAPTERS \\
\end{center}
\footnotesize \begin{itemize}
  \item[1.] Introductory.
  \item[2.] Geometric Co-ordinates
  \item[3.] The Doctrine of Limits.
  \item[4.] Newton and Some of his Successors.
  \item[5.] The Law of Sacrifice.
  \item[6.] Inspiration \emph{versus} Habit.
  \item[7.] Examples of Practical Application of the Mathematical Laws of Thought.
  \item[8.] The Sanity of True Genius.
  \item[ ] Appendix.
\end{itemize}\normalsize
\begin{center}
\rule{10cm}{1pt}
\textbf{London: C.W.\ DANIEL, 11 Cursitor Street, E.C.}
\end{center}}}
\end{center}

\newpage
\pagestyle{empty}
\small \pagenumbering{gobble}
\begin{verbatim}

End of the Project Gutenberg EBook of Philosophy and Fun of Algebra
by Mary Everest Boole

*** END OF THIS PROJECT GUTENBERG EBOOK PHILOSOPHY AND FUN OF ALGEBRA ***

***** This file should be named 13447-t.tex or 13447-t.zip *****
*****                    or 13447-pdf.pdf or 13447-pdf.zip *****
This and all associated files of various formats will be found in:
        http://www.gutenberg.net/1/3/4/4/13447/

Produced by Joshua Hutchinson, John Hagerson, and the Project
Gutenberg On-line Distributed Proofreaders. This book was produced
from images provided by Cornell University.


Updated editions will replace the previous one--the old editions
will be renamed.

Creating the works from public domain print editions means that no
one owns a United States copyright in these works, so the Foundation
(and you!) can copy and distribute it in the United States without
permission and without paying copyright royalties.  Special rules,
set forth in the General Terms of Use part of this license, apply to
copying and distributing Project Gutenberg-tm electronic works to
protect the PROJECT GUTENBERG-tm concept and trademark.  Project
Gutenberg is a registered trademark, and may not be used if you
charge for the eBooks, unless you receive specific permission.  If
you do not charge anything for copies of this eBook, complying with
the rules is very easy.  You may use this eBook for nearly any
purpose such as creation of derivative works, reports, performances
and research.  They may be modified and printed and given away--you
may do practically ANYTHING with public domain eBooks.
Redistribution is subject to the trademark license, especially
commercial redistribution.



*** START: FULL LICENSE ***

THE FULL PROJECT GUTENBERG LICENSE PLEASE READ THIS BEFORE YOU
DISTRIBUTE OR USE THIS WORK

To protect the Project Gutenberg-tm mission of promoting the free
distribution of electronic works, by using or distributing this work
(or any other work associated in any way with the phrase "Project
Gutenberg"), you agree to comply with all the terms of the Full
Project Gutenberg-tm License (available with this file or online at
http://gutenberg.net/license).


Section 1.  General Terms of Use and Redistributing Project
Gutenberg-tm electronic works

1.A.  By reading or using any part of this Project Gutenberg-tm
electronic work, you indicate that you have read, understand, agree
to and accept all the terms of this license and intellectual
property (trademark/copyright) agreement.  If you do not agree to
abide by all the terms of this agreement, you must cease using and
return or destroy all copies of Project Gutenberg-tm electronic
works in your possession. If you paid a fee for obtaining a copy of
or access to a Project Gutenberg-tm electronic work and you do not
agree to be bound by the terms of this agreement, you may obtain a
refund from the person or entity to whom you paid the fee as set
forth in paragraph 1.E.8.

1.B.  "Project Gutenberg" is a registered trademark.  It may only be
used on or associated in any way with an electronic work by people
who agree to be bound by the terms of this agreement.  There are a
few things that you can do with most Project Gutenberg-tm electronic
works even without complying with the full terms of this agreement.
See paragraph 1.C below.  There are a lot of things you can do with
Project Gutenberg-tm electronic works if you follow the terms of
this agreement and help preserve free future access to Project
Gutenberg-tm electronic works.  See paragraph 1.E below.

1.C.  The Project Gutenberg Literary Archive Foundation ("the
Foundation" or PGLAF), owns a compilation copyright in the
collection of Project Gutenberg-tm electronic works.  Nearly all the
individual works in the collection are in the public domain in the
United States.  If an individual work is in the public domain in the
United States and you are located in the United States, we do not
claim a right to prevent you from copying, distributing, performing,
displaying or creating derivative works based on the work as long as
all references to Project Gutenberg are removed.  Of course, we hope
that you will support the Project Gutenberg-tm mission of promoting
free access to electronic works by freely sharing Project
Gutenberg-tm works in compliance with the terms of this agreement
for keeping the Project Gutenberg-tm name associated with the work.
You can easily comply with the terms of this agreement by keeping
this work in the same format with its attached full Project
Gutenberg-tm License when you share it without charge with others.

1.D.  The copyright laws of the place where you are located also
govern what you can do with this work.  Copyright laws in most
countries are in a constant state of change.  If you are outside the
United States, check the laws of your country in addition to the
terms of this agreement before downloading, copying, displaying,
performing, distributing or creating derivative works based on this
work or any other Project Gutenberg-tm work.  The Foundation makes
no representations concerning the copyright status of any work in
any country outside the United States.

1.E.  Unless you have removed all references to Project Gutenberg:

1.E.1.  The following sentence, with active links to, or other
immediate access to, the full Project Gutenberg-tm License must
appear prominently whenever any copy of a Project Gutenberg-tm work
(any work on which the phrase "Project Gutenberg" appears, or with
which the phrase "Project Gutenberg" is associated) is accessed,
displayed, performed, viewed, copied or distributed:

This eBook is for the use of anyone anywhere at no cost and with
almost no restrictions whatsoever.  You may copy it, give it away or
re-use it under the terms of the Project Gutenberg License included
with this eBook or online at www.gutenberg.net

1.E.2.  If an individual Project Gutenberg-tm electronic work is
derived from the public domain (does not contain a notice indicating
that it is posted with permission of the copyright holder), the work
can be copied and distributed to anyone in the United States without
paying any fees or charges.  If you are redistributing or providing
access to a work with the phrase "Project Gutenberg" associated with
or appearing on the work, you must comply either with the
requirements of paragraphs 1.E.1 through 1.E.7 or obtain permission
for the use of the work and the Project Gutenberg-tm trademark as
set forth in paragraphs 1.E.8 or 1.E.9.

1.E.3.  If an individual Project Gutenberg-tm electronic work is
posted with the permission of the copyright holder, your use and
distribution must comply with both paragraphs 1.E.1 through 1.E.7
and any additional terms imposed by the copyright holder.
Additional terms will be linked to the Project Gutenberg-tm License
for all works posted with the permission of the copyright holder
found at the beginning of this work.

1.E.4.  Do not unlink or detach or remove the full Project
Gutenberg-tm License terms from this work, or any files containing a
part of this work or any other work associated with Project
Gutenberg-tm.

1.E.5.  Do not copy, display, perform, distribute or redistribute
this electronic work, or any part of this electronic work, without
prominently displaying the sentence set forth in paragraph 1.E.1
with active links or immediate access to the full terms of the
Project Gutenberg-tm License.

1.E.6.  You may convert to and distribute this work in any binary,
compressed, marked up, nonproprietary or proprietary form, including
any word processing or hypertext form.  However, if you provide
access to or distribute copies of a Project Gutenberg-tm work in a
format other than "Plain Vanilla ASCII" or other format used in the
official version posted on the official Project Gutenberg-tm web
site (www.gutenberg.net), you must, at no additional cost, fee or
expense to the user, provide a copy, a means of exporting a copy, or
a means of obtaining a copy upon request, of the work in its
original "Plain Vanilla ASCII" or other form.  Any alternate format
must include the full Project Gutenberg-tm License as specified in
paragraph 1.E.1.

1.E.7.  Do not charge a fee for access to, viewing, displaying,
performing, copying or distributing any Project Gutenberg-tm works
unless you comply with paragraph 1.E.8 or 1.E.9.

1.E.8.  You may charge a reasonable fee for copies of or providing
access to or distributing Project Gutenberg-tm electronic works
provided that

- You pay a royalty fee of 20% of the gross profits you derive from
     the use of Project Gutenberg-tm works calculated using the method
     you already use to calculate your applicable taxes.  The fee is
     owed to the owner of the Project Gutenberg-tm trademark, but he
     has agreed to donate royalties under this paragraph to the
     Project Gutenberg Literary Archive Foundation.  Royalty payments
     must be paid within 60 days following each date on which you
     prepare (or are legally required to prepare) your periodic tax
     returns.  Royalty payments should be clearly marked as such and
     sent to the Project Gutenberg Literary Archive Foundation at the
     address specified in Section 4, "Information about donations to
     the Project Gutenberg Literary Archive Foundation."

- You provide a full refund of any money paid by a user who notifies
     you in writing (or by e-mail) within 30 days of receipt that s/he
     does not agree to the terms of the full Project Gutenberg-tm
     License.  You must require such a user to return or
     destroy all copies of the works possessed in a physical medium
     and discontinue all use of and all access to other copies of
     Project Gutenberg-tm works.

- You provide, in accordance with paragraph 1.F.3, a full refund of
     any money paid for a work or a replacement copy, if a defect in
     the electronic work is discovered and reported to you within 90
     days of receipt of the work.

- You comply with all other terms of this agreement for free
     distribution of Project Gutenberg-tm works.

1.E.9.  If you wish to charge a fee or distribute a Project
Gutenberg-tm electronic work or group of works on different terms
than are set forth in this agreement, you must obtain permission in
writing from both the Project Gutenberg Literary Archive Foundation
and Michael Hart, the owner of the Project Gutenberg-tm trademark.
Contact the Foundation as set forth in Section 3 below.

1.F.

1.F.1.  Project Gutenberg volunteers and employees expend
considerable effort to identify, do copyright research on,
transcribe and proofread public domain works in creating the Project
Gutenberg-tm collection.  Despite these efforts, Project
Gutenberg-tm electronic works, and the medium on which they may be
stored, may contain "Defects," such as, but not limited to,
incomplete, inaccurate or corrupt data, transcription errors, a
copyright or other intellectual property infringement, a defective
or damaged disk or other medium, a computer virus, or computer codes
that damage or cannot be read by your equipment.

1.F.2.  LIMITED WARRANTY, DISCLAIMER OF DAMAGES - Except for the
"Right of Replacement or Refund" described in paragraph 1.F.3, the
Project Gutenberg Literary Archive Foundation, the owner of the
Project Gutenberg-tm trademark, and any other party distributing a
Project Gutenberg-tm electronic work under this agreement, disclaim
all liability to you for damages, costs and expenses, including
legal fees.  YOU AGREE THAT YOU HAVE NO REMEDIES FOR NEGLIGENCE,
STRICT LIABILITY, BREACH OF WARRANTY OR BREACH OF CONTRACT EXCEPT
THOSE PROVIDED IN PARAGRAPH F3.  YOU AGREE THAT THE FOUNDATION, THE
TRADEMARK OWNER, AND ANY DISTRIBUTOR UNDER THIS AGREEMENT WILL NOT
BE LIABLE TO YOU FOR ACTUAL, DIRECT, INDIRECT, CONSEQUENTIAL,
PUNITIVE OR INCIDENTAL DAMAGES EVEN IF YOU GIVE NOTICE OF THE
POSSIBILITY OF SUCH DAMAGE.

1.F.3.  LIMITED RIGHT OF REPLACEMENT OR REFUND - If you discover a
defect in this electronic work within 90 days of receiving it, you
can receive a refund of the money (if any) you paid for it by
sending a written explanation to the person you received the work
from.  If you received the work on a physical medium, you must
return the medium with your written explanation.  The person or
entity that provided you with the defective work may elect to
provide a replacement copy in lieu of a refund.  If you received the
work electronically, the person or entity providing it to you may
choose to give you a second opportunity to receive the work
electronically in lieu of a refund.  If the second copy is also
defective, you may demand a refund in writing without further
opportunities to fix the problem.

1.F.4.  Except for the limited right of replacement or refund set
forth in paragraph 1.F.3, this work is provided to you 'AS-IS', WITH
NO OTHER WARRANTIES OF ANY KIND, EXPRESS OR IMPLIED, INCLUDING BUT
NOT LIMITED TO WARRANTIES OF MERCHANTIBILITY OR FITNESS FOR ANY
PURPOSE.

1.F.5.  Some states do not allow disclaimers of certain implied
warranties or the exclusion or limitation of certain types of
damages. If any disclaimer or limitation set forth in this agreement
violates the law of the state applicable to this agreement, the
agreement shall be interpreted to make the maximum disclaimer or
limitation permitted by the applicable state law.  The invalidity or
unenforceability of any provision of this agreement shall not void
the remaining provisions.

1.F.6.  INDEMNITY - You agree to indemnify and hold the Foundation,
the trademark owner, any agent or employee of the Foundation, anyone
providing copies of Project Gutenberg-tm electronic works in
accordance with this agreement, and any volunteers associated with
the production, promotion and distribution of Project Gutenberg-tm
electronic works, harmless from all liability, costs and expenses,
including legal fees, that arise directly or indirectly from any of
the following which you do or cause to occur: (a) distribution of
this or any Project Gutenberg-tm work, (b) alteration, modification,
or additions or deletions to any Project Gutenberg-tm work, and (c)
any Defect you cause.


Section  2.  Information about the Mission of Project Gutenberg-tm

Project Gutenberg-tm is synonymous with the free distribution of
electronic works in formats readable by the widest variety of
computers including obsolete, old, middle-aged and new computers.
It exists because of the efforts of hundreds of volunteers and
donations from people in all walks of life.

Volunteers and financial support to provide volunteers with the
assistance they need, is critical to reaching Project Gutenberg-tm's
goals and ensuring that the Project Gutenberg-tm collection will
remain freely available for generations to come.  In 2001, the
Project Gutenberg Literary Archive Foundation was created to provide
a secure and permanent future for Project Gutenberg-tm and future
generations. To learn more about the Project Gutenberg Literary
Archive Foundation and how your efforts and donations can help, see
Sections 3 and 4 and the Foundation web page at
http://www.pglaf.org.


Section 3.  Information about the Project Gutenberg Literary Archive
Foundation

The Project Gutenberg Literary Archive Foundation is a non profit
501(c)(3) educational corporation organized under the laws of the
state of Mississippi and granted tax exempt status by the Internal
Revenue Service.  The Foundation's EIN or federal tax identification
number is 64-6221541.  Its 501(c)(3) letter is posted at
http://pglaf.org/fundraising.  Contributions to the Project
Gutenberg Literary Archive Foundation are tax deductible to the full
extent permitted by U.S. federal laws and your state's laws.

The Foundation's principal office is located at 4557 Melan Dr. S.
Fairbanks, AK, 99712., but its volunteers and employees are
scattered throughout numerous locations.  Its business office is
located at 809 North 1500 West, Salt Lake City, UT 84116, (801)
596-1887, email business@pglaf.org.  Email contact links and up to
date contact information can be found at the Foundation's web site
and official page at http://pglaf.org

For additional contact information:
     Dr. Gregory B. Newby
     Chief Executive and Director
     gbnewby@pglaf.org

Section 4.  Information about Donations to the Project Gutenberg
Literary Archive Foundation

Project Gutenberg-tm depends upon and cannot survive without wide
spread public support and donations to carry out its mission of
increasing the number of public domain and licensed works that can
be freely distributed in machine readable form accessible by the
widest array of equipment including outdated equipment.  Many small
donations ($1 to $5,000) are particularly important to maintaining
tax exempt status with the IRS.

The Foundation is committed to complying with the laws regulating
charities and charitable donations in all 50 states of the United
States.  Compliance requirements are not uniform and it takes a
considerable effort, much paperwork and many fees to meet and keep
up with these requirements.  We do not solicit donations in
locations where we have not received written confirmation of
compliance.  To SEND DONATIONS or determine the status of compliance
for any particular state visit http://pglaf.org

While we cannot and do not solicit contributions from states where
we have not met the solicitation requirements, we know of no
prohibition against accepting unsolicited donations from donors in
such states who approach us with offers to donate.

International donations are gratefully accepted, but we cannot make
any statements concerning tax treatment of donations received from
outside the United States.  U.S. laws alone swamp our small staff.

Please check the Project Gutenberg Web pages for current donation
methods and addresses.  Donations are accepted in a number of other
ways including including checks, online payments and credit card
donations.  To donate, please visit: http://pglaf.org/donate


Section 5.  General Information About Project Gutenberg-tm
electronic works.

Professor Michael S. Hart is the originator of the Project
Gutenberg-tm concept of a library of electronic works that could be
freely shared with anyone.  For thirty years, he produced and
distributed Project Gutenberg-tm eBooks with only a loose network of
volunteer support.

Project Gutenberg-tm eBooks are often created from several printed
editions, all of which are confirmed as Public Domain in the U.S.
unless a copyright notice is included.  Thus, we do not necessarily
keep eBooks in compliance with any particular paper edition.

Most people start at our Web site which has the main PG search
facility:

     http://www.gutenberg.net

This Web site includes information about Project Gutenberg-tm,
including how to make donations to the Project Gutenberg Literary
Archive Foundation, how to help produce our new eBooks, and how to
subscribe to our email newsletter to hear about new eBooks.
\end{verbatim}
\end{document}
---------------------------------------------------------
Below is appended the log from the most recent compile.
You may use it to compare against a log from a new
compile to help spot differences.
---------------------------------------------------------
This is pdfeTeX, Version 3.141592-1.21a-2.2 (MiKTeX 2.4) (preloaded format=latex 2005.4.4)  4 DEC 2005 12:14
entering extended mode
**13447-t
(13447-t.tex
LaTeX2e <2003/12/01>
Babel <v3.8a> and hyphenation patterns for english, french, german, ngerman, du
mylang, nohyphenation, loaded.
(C:\texmf\tex\latex\base\book.cls
Document Class: book 2004/02/16 v1.4f Standard LaTeX document class
(C:\texmf\tex\latex\base\bk10.clo
File: bk10.clo 2004/02/16 v1.4f Standard LaTeX file (size option)
)
\c@part=\count79
\c@chapter=\count80
\c@section=\count81
\c@subsection=\count82
\c@subsubsection=\count83
\c@paragraph=\count84
\c@subparagraph=\count85
\c@figure=\count86
\c@table=\count87
\abovecaptionskip=\skip41
\belowcaptionskip=\skip42
\bibindent=\dimen102
) (C:\texmf\tex\latex\base\inputenc.sty
Package: inputenc 2004/02/05 v1.0d Input encoding file

(C:\texmf\tex\latex\base\latin1.def
File: latin1.def 2004/02/05 v1.0d Input encoding file
)) (C:\texmf\tex\latex\amsmath\amsmath.sty
Package: amsmath 2000/07/18 v2.13 AMS math features
\@mathmargin=\skip43

For additional information on amsmath, use the `?' option.
(C:\texmf\tex\latex\amsmath\amstext.sty
Package: amstext 2000/06/29 v2.01
 (C:\texmf\tex\latex\amsmath\amsgen.sty
File: amsgen.sty 1999/11/30 v2.0
\@emptytoks=\toks14
\ex@=\dimen103
)) (C:\texmf\tex\latex\amsmath\amsbsy.sty
Package: amsbsy 1999/11/29 v1.2d
\pmbraise@=\dimen104
)
(C:\texmf\tex\latex\amsmath\amsopn.sty
Package: amsopn 1999/12/14 v2.01 operator names
)
\inf@bad=\count88
LaTeX Info: Redefining \frac on input line 211.
\uproot@=\count89
\leftroot@=\count90
LaTeX Info: Redefining \overline on input line 307.
\classnum@=\count91
\DOTSCASE@=\count92
LaTeX Info: Redefining \ldots on input line 379.
LaTeX Info: Redefining \dots on input line 382.
LaTeX Info: Redefining \cdots on input line 467.
\Mathstrutbox@=\box26
\strutbox@=\box27
\big@size=\dimen105
LaTeX Font Info:    Redeclaring font encoding OML on input line 567.
LaTeX Font Info:    Redeclaring font encoding OMS on input line 568.
\macc@depth=\count93
\c@MaxMatrixCols=\count94
\dotsspace@=\muskip10
\c@parentequation=\count95
\dspbrk@lvl=\count96
\tag@help=\toks15
\row@=\count97
\column@=\count98
\maxfields@=\count99
\andhelp@=\toks16
\eqnshift@=\dimen106
\alignsep@=\dimen107
\tagshift@=\dimen108
\tagwidth@=\dimen109
\totwidth@=\dimen110
\lineht@=\dimen111
\@envbody=\toks17
\multlinegap=\skip44
\multlinetaggap=\skip45
\mathdisplay@stack=\toks18
LaTeX Info: Redefining \[ on input line 2666.
LaTeX Info: Redefining \] on input line 2667.
)
(C:\texmf\tex\latex\graphics\graphicx.sty
Package: graphicx 1999/02/16 v1.0f Enhanced LaTeX Graphics (DPC,SPQR)

(C:\texmf\tex\latex\graphics\keyval.sty
Package: keyval 1999/03/16 v1.13 key=value parser (DPC)
\KV@toks@=\toks19
)
(C:\texmf\tex\latex\graphics\graphics.sty
Package: graphics 2001/07/07 v1.0n Standard LaTeX Graphics (DPC,SPQR)
 (C:\texmf\tex\latex\graphics\trig.sty
Package: trig 1999/03/16 v1.09 sin cos tan (DPC)
) (C:\texmf\tex\latex\00miktex\graphics.cfg
File: graphics.cfg 2003/03/12 v1.1 MiKTeX 'graphics' configuration
)
Package graphics Info: Driver file: pdftex.def on input line 80.

(C:\texmf\tex\latex\graphics\pdftex.def
File: pdftex.def 2002/06/19 v0.03k graphics/color for pdftex
\Gread@gobject=\count100
))
\Gin@req@height=\dimen112
\Gin@req@width=\dimen113
)
(C:\texmf\tex\latex\yfonts\yfonts.sty
Package: yfonts 2003/01/08 v1.3 (WaS)
) (13447-t.aux)
LaTeX Font Info:    Checking defaults for OML/cmm/m/it on input line 60.
LaTeX Font Info:    ... okay on input line 60.
LaTeX Font Info:    Checking defaults for T1/cmr/m/n on input line 60.
LaTeX Font Info:    ... okay on input line 60.
LaTeX Font Info:    Checking defaults for OT1/cmr/m/n on input line 60.
LaTeX Font Info:    ... okay on input line 60.
LaTeX Font Info:    Checking defaults for OMS/cmsy/m/n on input line 60.
LaTeX Font Info:    ... okay on input line 60.
LaTeX Font Info:    Checking defaults for OMX/cmex/m/n on input line 60.
LaTeX Font Info:    ... okay on input line 60.
LaTeX Font Info:    Checking defaults for U/cmr/m/n on input line 60.
LaTeX Font Info:    ... okay on input line 60.
LaTeX Font Info:    Checking defaults for LY/yfrak/m/n on input line 60.
LaTeX Font Info:    ... okay on input line 60.
LaTeX Font Info:    Checking defaults for LYG/ygoth/m/n on input line 60.
LaTeX Font Info:    ... okay on input line 60.

(C:\texmf\tex\context\base\supp-pdf.tex (C:\texmf\tex\context\base\supp-mis.tex
loading : Context Support Macros / Miscellaneous (2004.10.26)
\protectiondepth=\count101
\scratchcounter=\count102
\scratchtoks=\toks20
\scratchdimen=\dimen114
\scratchskip=\skip46
\scratchmuskip=\muskip11
\scratchbox=\box28
\scratchread=\read1
\scratchwrite=\write3
\zeropoint=\dimen115
\onepoint=\dimen116
\onebasepoint=\dimen117
\minusone=\count103
\thousandpoint=\dimen118
\onerealpoint=\dimen119
\emptytoks=\toks21
\nextbox=\box29
\nextdepth=\dimen120
\everyline=\toks22
\!!counta=\count104
\!!countb=\count105
\recursecounter=\count106
)
loading : Context Support Macros / PDF (2004.03.26)
\nofMPsegments=\count107
\nofMParguments=\count108
\MPscratchCnt=\count109
\MPscratchDim=\dimen121
\MPnumerator=\count110
\everyMPtoPDFconversion=\toks23
)
Overfull \hbox (9.37088pt too wide) in paragraph at lines 99--99
[]\OT1/cmtt/m/n/9 *** START OF THIS PROJECT GUTENBERG EBOOK PHILOSOPHY AND FUN 
OF ALGEBRA ***[] 
 []

[1

{psfonts.map}] [1

] <images/cornell.png, id=23, 585.8688pt x 510.708pt>
File: images/cornell.png Graphic file (type png)

<use images/cornell.png> [2 <images/cornell.png>] [3] [4] (13447-t.toc)
\tf@toc=\write4

[5

]
Chapter 1.
[1


] [2] [3]
Chapter 2.
[4

] [5]
Chapter 3.
[6

] [7]
Chapter 4.
[8

] [9]
Chapter 5.
[10

] [11]
Chapter 6.
[12

] [13] [14]
Chapter 7.
[15

] [16]
Overfull \hbox (3.4451pt too wide) in paragraph at lines 887--889
[]\OT1/cmr/m/n/10 Be rather par-tic-u-lar not to eat any-thing ei-ther in-di-ge
stible or highly flavoured. 
 []

[17] [18]
Chapter 8.
[19

] [20]
Chapter 9.
[21

] [22]
Chapter 10.
[23

] [24] [25]
Chapter 11.
[26

] [27]
Chapter 12.
[28

]
Chapter 13.
[29

] [30]
Chapter 14.
[31

] [32]
Chapter 15.
[33

]
Chapter 16.
[34

] [35]
Chapter 17.
[36

] [37]
Chapter 18.
[38

]
Overfull \vbox (63.52219pt too high) has occurred while \output is active []


[39] [40] [41] [42] [1] [2] [3] [4] [5] [6] [7] [8] (13447-t.aux)

 *File List*
    book.cls    2004/02/16 v1.4f Standard LaTeX document class
    bk10.clo    2004/02/16 v1.4f Standard LaTeX file (size option)
inputenc.sty    2004/02/05 v1.0d Input encoding file
  latin1.def    2004/02/05 v1.0d Input encoding file
 amsmath.sty    2000/07/18 v2.13 AMS math features
 amstext.sty    2000/06/29 v2.01
  amsgen.sty    1999/11/30 v2.0
  amsbsy.sty    1999/11/29 v1.2d
  amsopn.sty    1999/12/14 v2.01 operator names
graphicx.sty    1999/02/16 v1.0f Enhanced LaTeX Graphics (DPC,SPQR)
  keyval.sty    1999/03/16 v1.13 key=value parser (DPC)
graphics.sty    2001/07/07 v1.0n Standard LaTeX Graphics (DPC,SPQR)
    trig.sty    1999/03/16 v1.09 sin cos tan (DPC)
graphics.cfg    2003/03/12 v1.1 MiKTeX 'graphics' configuration
  pdftex.def    2002/06/19 v0.03k graphics/color for pdftex
  yfonts.sty    2003/01/08 v1.3 (WaS)
supp-pdf.tex
images/cornell.png
 ***********

 ) 
Here is how much of TeX's memory you used:
 1756 strings out of 95512
 19308 string characters out of 1189449
 83894 words of memory out of 1083062
 4807 multiletter control sequences out of 60000
 14026 words of font info for 50 fonts, out of 500000 for 1000
 14 hyphenation exceptions out of 607
 27i,8n,23p,220b,273s stack positions out of 1500i,500n,5000p,200000b,32768s
PDF statistics:
 230 PDF objects out of 300000
 0 named destinations out of 300000
 6 words of extra memory for PDF output out of 65536
<C:\texmf\fonts\type1\bluesky\cm\cmmi12.pfb><C:\texmf\fonts\type1\bluesky\cm\
cmti8.pfb><C:\texmf\fonts\type1\bluesky\cm\cmr6.pfb><C:\texmf\fonts\type1\blues
ky\cm\cmr7.pfb><C:\texmf\fonts\type1\bluesky\cm\cmsl10.pfb><C:\texmf\fonts\type
1\bluesky\cm\cmsy10.pfb><C:\texmf\fonts\type1\bluesky\cm\cmmi10.pfb><C:\texmf\f
onts\type1\bluesky\cm\cmbx12.pfb><C:\texmf\fonts\type1\bluesky\cm\cmti10.pfb><C
:\texmf\fonts\type1\public\gothict1\ygoth.pfb><C:\texmf\fonts\type1\bluesky\cm\
cmcsc10.pfb><C:\texmf\fonts\type1\bluesky\cm\cmbx10.pfb><C:\texmf\fonts\type1\b
luesky\cm\cmr8.pfb><C:\texmf\fonts\type1\bluesky\cm\cmr12.pfb><C:\texmf\fonts\t
ype1\bluesky\cm\cmr17.pfb><C:\texmf\fonts\type1\bluesky\cm\cmr10.pfb><C:\texmf\
fonts\type1\bluesky\cm\cmtt9.pfb>
Output written on 13447-t.pdf (56 pages, 275530 bytes).

